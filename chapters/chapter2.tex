\chapter{Ensembles et applications}
\section{Ensembles}

\begin{definitionbox}
    \begin{definition}[Définition intuitive d'un ensemble]
	Un ensemble E est une collection d'objets appelés éléments.
	Si E contient un élément $x$, on dit que $x$ appartient à E, noté $x \in E$ 
\end{definition}
\end{definitionbox}

\begin{definitionbox}
    \begin{definition}[Ensemble vide]
	L'ensemble vide noté $\emptyset$ est l'ensemble ne contenant aucun élément.
\end{definition}
\end{definitionbox}

\begin{definitionbox}
    \begin{definition}[Inclusion]
	\begin{align*}
		&\text{Un ensemble F est inclus dans un ensemble E } \iff \forall x \in F,\ x \in E. \\
		&\text{On note : } F \subset E \text{ On dit aussi que F est un sous-ensemble, une partie de E}
	\end{align*}
\end{definition}
\end{definitionbox}

\begin{definitionbox}
    \begin{definition}[Egalité d'ensembles]
	\begin{align*}
		\text{Deux ensembles E et F sont égaux} \iff E \subset F \text{ et } F \subset E
	\end{align*} 
\end{definition}
\end{definitionbox}

\begin{definitionbox}
\begin{definition}[Singleton]
	Un singleton est un ensemble de ne contenant qu'un seul élément (noté entre accolades).
\end{definition}
\end{definitionbox}

\begin{definitionbox}
    \begin{definition}[Réunion d'ensembles]
	Soient E et F deux ensembles.
	\begin{align*}
		E \cup F \text{ (lu E union F) } = \{\forall x,\ x \in E \text{ ou } x \in F \}
	\end{align*}
\end{definition}
\end{definitionbox}

\begin{definitionbox}
    \begin{definition}[Intersection d'ensembles]
	Soient E et F deux ensembles.
	\begin{align*}
		E \cap F \text{ (lu E inter F) } = \{\forall x,\ x \in E \text{ et } x \in F \}
	\end{align*}
\end{definition}
\end{definitionbox}

\begin{propositionbox}
    \begin{proposition}[Propriétés sur les ensembles] Soient A, B, C, E des ensembles
	\begin{enumerate}
		\item Associativité :
		\begin{align*}
            \boxshadow{A \cup (B \cup C) = (A \cup B) \cup C} \\
            \boxshadow{A \cap (B \cap C) = (A \cap B) \cap C}
		\end{align*}
		\item Elément neutre : 
		\begin{align*}
            \boxshadow{A \cup \emptyset = A} \\
            \boxshadow{A \cap A = A} 
		\end{align*}
		\item Intersection d'un ensemble et d'une partie :
		\begin{align*}
            \boxshadow{A \subset E \iff \ A \cap E = E \cap A = A}
		\end{align*}
		\item Commutativité :
		\begin{align*}
            \boxshadow{A \cup B = B \cup A} \\
            \boxshadow{A \cap B = B \cap A}	
		\end{align*}
		\item Distributivité :
		\begin{align*}
            \boxshadow{A \cup (B \cap C) = (A \cup B) \cap (A \cup C)} \\
            \boxshadow{A \cap (B \cup C) = (A \cap B) \cup (A \cap C)}	
		\end{align*}
	\end{enumerate}
\end{proposition}
\end{propositionbox}

\begin{definitionbox}
    \begin{definition}[Complémentaire d'un ensemble]
	\begin{align*}
        \boxshadow{E \backslash F = \{\forall x,\ x \in E \text{ et } x \notin F \}}	
	\end{align*}
\end{definition}
\end{definitionbox}

\begin{leftstroke}
\begin{remarque}Soient E, F des ensembles.
	\begin{itemize}
		\item $(E \backslash F) \subset E$
		\item $(E \backslash F) \cap F = \emptyset$
		\item $E \backslash F \neq F \backslash E$
	\end{itemize}
\end{remarque}
\end{leftstroke}

\begin{leftstroke}
\begin{remarque}Soient E et A des ensembles.
	\begin{align*}
		&A \subset E \\
		&A^C = E \backslash A \\
		&(A^C)^C = A
	\end{align*}
\end{remarque}
\end{leftstroke}

\begin{propositionbox}
    \begin{proposition}[Lois de Morgan] Soient A et B des ensembles.
	\begin{enumerate}
        \item \label{prop10_1} $\boxshadow{(A \cup B)^C = A^C \cap B^C}$
        \item \label{prop10_2} $\boxshadow{(A \cap B)^C = A^C \cup B^C}$
	\end{enumerate}
\end{proposition}
\end{propositionbox}
\begin{proof}
	\ref{prop10_1} \\
	Soient A et B des ensembles et $x$ un élément quelconque.
	\\
	\framebox{$\subset$}
	Par définition du complémentaire : 
	\begin{align*}
		x \in (A \cup B)^C \iff x \notin (A \cup B)
	\end{align*}
	$x \notin A$ car $A \subset (A \cup B)$ ce qui impliquerait que $x \in (A \cup B)$ et donc il y aurait une contradiction. On obtient une contradiction similaire si on suppose que $x \in B$. Ainsi on a $x \in A^C \text{ et } x \in B^C$, donc par la définition de l'intersection on a :
	\begin{align*}
		x \in (A^C \cap B^C)
	\end{align*}
	d'où :
	\begin{align*}
		(A \cup B)^C \subset (A^C \cap B^C)
	\end{align*}
	\framebox{$\supset$} Par définition de l'intersection :
	\begin{align*}
		x \in (A^C \cap B^C) &\iff x \in A^C \text{ et } x \in B^C \\
		&\iff x \notin A \text{ et } x \notin B \\
		&\iff x \in (A \cup B)^C
	\end{align*}
	d'où :
	\begin{align*}
		(A^C \cap B^C) \subset (A \cup B)^C
	\end{align*}
	Ainsi : 
	\begin{align*}
		(A \cup B)^C = A^C \cap B^C
	\end{align*}
\end{proof}

\begin{proof}
	\ref{prop10_2} 
	\\
	Soient A et B des ensembles et $x$ un élément quelconque. \\
	\framebox{$\subset$} Par définition du complémentaire :
	\begin{align*}
		x \in (A \cap B)^C &\iff x \notin (A \cap B) \\ 
		&\iff x \notin A \text{ et } x \notin B \\
		&\iff x \in A^C \text{ et } x \in B^C \\
		&\iff x \in (A^C \cap B^C)
	\end{align*}
	Sachant que :
	\begin{align*}
		(A^C \cap B^C) \subset (A^C \cup B^C)
	\end{align*}
	On a :
	\begin{align*}
		x \in (A^C \cap B^C) \implies x \in (A^C \cup B^C)
	\end{align*}
	d'où :
	\begin{align*}
		(A \cap B)^C \subset (A^C \cup B^C)
	\end{align*}
	\framebox{$\supset$} Par définition de la réunion :
	\begin{align*}
		x \in (A^C \cup B^C) &\iff x \in A^C \text{ ou } x \in B^C \\
		&\iff x \notin A \text{ ou } x \notin B \\
		&\iff x \notin (A \cap B) \\
		&\iff x \in (A \cap B)^C
	\end{align*}
	Ainsi : 
	\begin{align*}
		(A^C \cap B^C) \subset (A \cup B)^C
	\end{align*}
	Donc :
	\begin{align*}
		(A \cap B)^C = A^C \cup B^C
	\end{align*}
\end{proof}

\begin{definitionbox}
    \begin{definition}[Produit cartésien]
	Soient E et F des ensembles
	\begin{align*}
        &- \boxshadow{E \times F = \{(x, y),\ x \in E,\ y \in F\}} \\
        &- \boxshadow{E \times E = E^2} \\
        &- \boxshadow{E \times E \times E = E^3} 
	\end{align*}
\end{definition}
\end{definitionbox}
\section{Applications}
\begin{definitionbox}
    \begin{definition}[Application]
	Soient E et F deux ensembles. $f:E \to F$ est une application si pour chaque $x \in E$, on associe un élément de F noté $f(x)$
\end{definition}
\end{definitionbox}

\begin{definitionbox}
    \begin{definition}[Injectivité]
	Soit $f:E \to F$, on dit que $f$ est injective si pour chaque élément de F, il y a au plus un élément de E qui y est associé. Autrement dit :
	\begin{align*}
        \boxshadow{\text{f injective} \iff \{\forall (x_1, x_2) \in E^2, f(x_1) = f(x_2) \implies x_1 = x_2\}}	
	\end{align*}
\end{definition}
\end{definitionbox}

\begin{definitionbox}
    \begin{definition}[Surjectivité]
	Soit $f:E \to F$, on dit que $f$ est surjective si pour chaque élément de F, il y a au moins un élément de E qui y est associé.
	Autrement dit :
	\begin{align*}
        \boxshadow{\text{f surjective} \iff \{\forall y \in F, \exists x \in E, y = f(x)\}}	
	\end{align*}
\end{definition}
\end{definitionbox}

\begin{definitionbox}
    \begin{definition}[Bijectivité]
	Soit $f:E \to F$, on dit que $f$ est bijective si elle est injective et surjective, c'est-à-dire que pour chaque élément de F, il y a exactement un élément de E qui y est associé.
	Autrement dit :
	\begin{align*}
        \boxshadow{\text{f bijective} \iff \{\forall y \in F, \exists! x \in E, y = f(x)\}}	
	\end{align*}
\end{definition}
\end{definitionbox}

\begin{definitionbox}
    \begin{definition}[Ensemble fini]
    Un ensemble E est un ensemble fini non-vide si et seulement si pour tout entier $n \geq 1$, il existe une application bijective de $\{1, 2, \ldots, n\}$ dans E.
\end{definition}
\end{definitionbox}

\begin{definitionbox}
    \begin{definition}[Fonction réciproque]
	Soient E et F deux ensembles. Supposons que $f:E \to F$ est une application bijective. On peut définir l'application 
	\begin{align*}
	    \boxshadow{f^{-1} : 
		\begin{cases}
			F &\to E \\
			y &\mapsto x
    \end{cases}}
	\end{align*}
	comme étant la réciproque de $f$.
\end{definition}
\end{definitionbox}

\begin{definitionbox}
    \begin{definition}[Composition]
	Soient $f$ et $g$ deux applications telles que :
	$f:E \to F$ et $g:F \to G$ on a l'application $g \circ f : E \to G$ qui est définie comme étant la composée de $f$ et de $g$.
\end{definition}
\end{definitionbox}

\begin{definitionbox}
    \begin{definition}[Image directe et image réciproque]
	Soient $f:E \to F$ une application, A une partie de E et B une partie de F. Nous avons :
	\begin{align*}
        \boxshadow{f(A) = \{f(x), x \in A\} \text{ : image directe}} \\
        \boxshadow{f^{-1}(B) = \{x \in E, f(x) \in B\} \text{ : image réciproque}}	
	\end{align*}
\end{definition}
\end{definitionbox}

\begin{propositionbox}
    \begin{proposition}[Propriétés sur les images directes et réciproques]
	Soient $f:E \to F$ une application et A, B des parties de F.
	\begin{enumerate}
        \item \label{prop_img_1} $\boxshadow{f^{-1}(F \backslash A) = E \backslash f^{-1}(A)}$
        \item \label{prop_img_2} $\boxshadow{f^{-1}(A \cup B) = f^{-1}(A) \cup f^{-1}(B)}$
        \item \label{prop_img_3} $\boxshadow{f^{-1}(A \cap B) = f^{-1}(A) \cap f^{-1}(B)}$
        \item \label{prop_img_4} $\boxshadow{f(A \cup B) = f(A) \cup f(B)}$
        \item \label{prop_img_5} $\boxshadow{f(A \cap B) \subset f(A) \cap f(B)}$
	\end{enumerate}
\end{proposition}
\end{propositionbox}

\begin{proof}
	\ref{prop_img_5} : $f(A \cap B) \subset f(A) \cap f(B)$ \\
	\par \noindent Soit $y \in f(A \cap B) = \{f(x), x \in A \cap B\}$, par définition : $\exists x \in A \cap B, y = f(x)$
	\begin{align*}
		x \in A \cap B &\iff x \in A \wedge x \in B \\
		x \in A &\implies y = f(x) \subset f(A) \\
		x \in B &\implies y = f(x) \subset f(B) 
	\end{align*}
	d'où $y \in f(A) \cap f(B)$
\end{proof} 

\begin{leftstroke}
\begin{remarque}
	\begin{align*}
		f(A \cap B) \neq f(A) \cap f(B)
	\end{align*}
\end{remarque}
\end{leftstroke}
