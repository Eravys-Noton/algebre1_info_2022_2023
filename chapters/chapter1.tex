\chapter{Calcul Algébrique}
\section{Point sur les ensembles de nombres}

\begin{definitionbox}
\begin{definition}[Ensemble des nombres entiers naturels]
	\begin{align*}
        \boxshadow{\N = \{0; 1; ...\}}
    \end{align*}
\end{definition}
\end{definitionbox}

\begin{definitionbox}
    \begin{definition}[Ensemble des nombres entiers relatifs]
	\begin{align*}
        \boxshadow{\Z = \{...; -1; 0; 1; ...\}}
    \end{align*}
\end{definition}
\end{definitionbox}

\begin{definitionbox}
    \begin{definition}[Ensemble des nombres rationnels]
	\begin{align*}
        \boxshadow{\Q = \left\{\frac{a}{b} \mid a \in \Z, b \in \N^* \right\}}
	\end{align*}
\end{definition}
\end{definitionbox}

\begin{definitionbox}
\begin{definition}{Ensemble des nombres réels}
	\begin{align*}
        \boxshadow{\R = ]-\infty; +\infty[}	
	\end{align*}
\end{definition}
\end{definitionbox}

\subsection{Axiomatique}
Ici $\K$ désigne soit $\N$, soit $\Z$, soit $\Q$, soit $\R$

\begin{propositionbox}
\begin{proposition}{Loi de composition +}
	\begin{enumerate}
		\item Associativité :
		\begin{align*}
            \boxshadow{\forall(a, b, c) \in \K^3,\ a + (b + c) = (a + b) + c}	
		\end{align*}
		\item Commutativité :
		\begin{align*}
            \boxshadow{\forall(a, b) \in \K^2,\ a + b = b + a}	
		\end{align*}
		\item Existence d'un élément neutre : 
		\begin{align*}
            \boxshadow{\forall a \in \K,\ a + 0 = a}	
		\end{align*}
		\item Symétrie :
		\begin{align*}
            \boxshadow{\forall(a, a') \in \K^2,\ a + a' = 0 \textnormal{ avec } a' = -a}	
		\end{align*}
		Remarque : Cette propriété ne s'applique pas dans $\N$
	\end{enumerate}
\end{proposition}
\end{propositionbox}

\begin{propositionbox}
\begin{proposition}{Loi de composition $\cdot$}
	\begin{enumerate}
		\item Associativité :
		\begin{align*}
            \boxshadow{\forall (a, b, c) \in \K^3,\ (a \cdot b) \cdot c = a \cdot (b \cdot c)}	
		\end{align*}
		\item Commutativité :
		\begin{align*}
            \boxshadow{\forall (a, b) \in \K^3,\ a \cdot b = b \cdot a}	
		\end{align*}
		\item Existence d'un élément neutre : $\forall a \in \K$
		\begin{align*}
              & \boxshadow{a \cdot 1 = a} \\
              & \boxshadow{a \cdot 0 = 0}
		\end{align*}
		\item Distributivité : $\forall (a, b, c) \in \K^3$
		\begin{align*}
             &\boxshadow{a \cdot (b + c) = a \cdot b + a \cdot c} \\
             & \boxshadow{(a + b) \cdot c = a \cdot c + b \cdot c}
		\end{align*}
	\end{enumerate}
\end{proposition}
\end{propositionbox}

\section{Opérations sur les fractions}

\begin{propositionbox}
\begin{proposition}{Addition sur les fractions}
	\begin{align*}
        \boxshadow{\forall (a, b, c, d) \in \Z^4,\ \frac{a}{b} + \frac{c}{d} = \frac{ad + bc}{bd}}	
	\end{align*}
\end{proposition}
\end{propositionbox}

\begin{proof}~
	\\
	$\forall (a, b, c, d, a', b', c', d') \in \Z^8$
	
	\noindent D'après la proposition on a :
	\begin{align*}
		\frac{a}{b} + \frac{c}{d} = \frac{ad + bc}{bd}
	\end{align*}
	et :
	\begin{align*}
		\frac{a'}{b'} + \frac{c'}{d'} = \frac{a'd' + b'c'}{b'd'}
	\end{align*}
	Montrons que : 
	\begin{align*}
		\frac{ad + bc}{bd} = \frac{a'd' + b'c'}{b'd'}
	\end{align*}
	On suppose que : 
	\begin{align*}
		\frac{a}{b} = \frac{a'}{b'} \iff a'b = ab' \\
		\frac{c}{d} = \frac{c'}{d'} \iff c'd = cd'
	\end{align*}
	On aurait donc :
	\begin{align*}
		&\frac{ad + bc}{bd} = \frac{a'd' + b'c'}{b'd'} \\
		\iff &(ad + bc)b'd' = bd(a'd' + b'c') \\
		\iff &(ad + bc)b'd' - bd(a'd' + b'c') = 0
	\end{align*}
	\begin{align*}
		(ad + bc)b'd' - bd(a'd' + b'c') &= (adb'd' + bcb'd') - (bda'd' + bdb'c')\\
		&= adb'd' + bcb'd' - bda'd' - bdb'c' \\
		&= adb'd' - a'd'bd + bcb'd' - b'c'bd \\
		&= ab'dd' - a'bdd' + cd'bb' - c'dbb' \\
		&= (ab' - a'b)dd' + (cd' - c'd)dd'
	\end{align*}
	D'après l'hypothèse de départ :
	\begin{align*}
		ab' = a'b \iff ab' - a'b = 0 \\
		cd' = c'd \iff cd' - c'd = 0
	\end{align*}
	Donc : 
	\begin{align*}
		(\underbrace{ab' - a'b}_{0})dd' + (\underbrace{cd' - c'd}_0)dd' = 0
	\end{align*}
	On obtient alors :
	\begin{align*}
		(ad + bc)b'd' - bd(a'd' + b'c') = 0
	\end{align*}
\end{proof}

\begin{propositionbox}
\begin{proposition}[Multiplication sur les fractions]
	\begin{align*}
        \boxshadow{\forall (a, b, c, d) \in \Z^4,\ \frac{a}{b} \times \frac{c}{d} = \frac{ac}{bd}}	
	\end{align*}
\end{proposition}
\end{propositionbox}

\begin{proof}~
	\\
	$\forall (a, b, c, d, a', b', c', d') \in \Z^8$\\
	D'après la proposition on a :
	\begin{align*}
		\frac{a}{b} \cdot \frac{c}{d} = \frac{ac}{bd}
	\end{align*}
	et :
	\begin{align*}
		\frac{a'}{b'} \cdot \frac{c'}{d'} = \frac{a'c'}{b'd'}
	\end{align*}
	Montrons que :
	\begin{align*}
		\frac{ac}{bd} = \frac{a'c'}{b'd'}
	\end{align*}
	On suppose que : 
	\begin{align*}
		\frac{a}{b} = \frac{a'}{b'} \iff a'b = ab' \\
		\frac{c}{d} = \frac{c'}{d'} \iff c'd = cd'
	\end{align*}
	On aurait donc :
	\begin{align*}
		&\frac{ac}{bd} = \frac{a'c'}{b'd'} \\
		\iff &acb'd' = bda'c' \\
		\iff &acb'd' - bda'c' = 0
	\end{align*}
	\begin{align*}
		acb'd' - bda'c' &= (ab')(cd') - (a'b)(c'd) \\
		&= (ab')(cd') - (a'b)(cd') + (a'b)(cd') - (a'b)(c'd) \\
		&= (ab' - a'b)(cd') + (cd' - c'd)(a'b)
	\end{align*}
	D'après l'hypothèse de départ :
	\begin{align*}
		ab' = a'b \iff ab' - a'b = 0 \\
		cd' = c'd \iff cd' - c'd = 0
	\end{align*}
	Donc :
	\begin{align*}
		(\underbrace{ab' - a'b}_0)(cd') + (\underbrace{cd' - c'd}_0)(a'b) = 0
	\end{align*}
	On obtient alors :
	\begin{align*}
		acb'd' - bda'c' = 0
	\end{align*}
\end{proof}
\section{Sommes}

\begin{definitionbox}
    \begin{definition}[Définition de la somme]
	$\forall m, n \in \N \textnormal{ avec } m \leq n \textnormal{ et } a_k \in \R, \ m \leq k \leq n$
	\begin{align*}
        \boxshadow{\sum_{k = m}^{n}a_k = a_m + a_{m+1} + \cdots + a_n}	
	\end{align*}
\end{definition}
\end{definitionbox}

\begin{leftstroke}
\begin{remarque}
	L'indice de sommation est important car :
	\begin{align*}
		\sum_{k = m}^{n}a_l = \underbrace{a_l + a_l + \cdots + a_l}_{n - m + 1 \textnormal{termes}} = (n - m + 1)a_l
	\end{align*}
\end{remarque}
\end{leftstroke}

\begin{propositionbox}
    \begin{proposition}[Linéarité de la somme]
	$\forall m, n \in \N \textnormal{ avec } m \leq n$ et $a_k, b_k \in \R, \ m \leq k \leq n$
	\begin{align*}
        \boxshadow{\sum_{k = m}^{n} (a_k + b_k) = \sum_{k=m}^{n}a_k + \sum_{k=m}^{n}b_k}	
	\end{align*}
\end{proposition}
\end{propositionbox}
\begin{proof}
	$\forall m, n \in \N \textnormal{ avec } m \leq n \textnormal{ et } a_k, b_k \in \R, \ m \leq k \leq n$
	\begin{align*}
		\sum_{k = m}^{n} (a_k + b_k) &= (a_m + b_m) + (a_{m+1} + b_{m+1}) + \cdots (a_n + b_n)\\
		&= (a_m + a_{m+1} + \cdots + a_n) + (b_m + b_{m+1} + \cdots + b_n) \\
		&= \sum_{k=m}^{n}a_k + \sum_{k=m}^{n}b_k
	\end{align*}
\end{proof}

\begin{propositionbox}
    \begin{proposition}[Linéarité de la multiplication de la somme par une constante]
	$\forall m, n \in \N \textnormal{ avec } m \leq n \textnormal{ et } a_k, \lambda \in \R, \ m \leq k \leq n$
	\begin{align*}
        \boxshadow{\sum_{k = m}^{n} (\lambda a_k) = \lambda \sum_{k = m}^{n} a_k}	
	\end{align*}
\end{proposition}
\end{propositionbox}

\begin{proof}
	$\forall m, n \in \N \textnormal{ avec } m \leq n \textnormal{ et } a_k, \lambda \in \R, \ m \leq k \leq n$
	\begin{align*}
		\sum_{k = m}^{n} (\lambda a_k) &=  \lambda a_m + \lambda a_{m+1} + \cdots + \lambda a_n \\
		&= \lambda (a_m + a_{m+1} + \cdots + a_n)
	\end{align*}
\end{proof}
\subsection{Quelques sommes importantes}
\begin{enumerate}
	\item \label{1.3.1_sum1} $\displaystyle{\sum_{k = 1}^{n} k = \frac{n(n+1)}{2}}$ avec $n \in \N$
	\item \label{1.3.1_sum2} $\displaystyle{\sum_{k = 1}^{n}\left(a + \left(k - 1\right)d\right) = \frac{1}{2}n\left(2a + \left(n - 1\right)d\right)}$ avec $n \in \N \text{ et } a, d \in \R$
	\item \label{1.3.1_sum3} $\displaystyle{\sum_{k = 0}^{n-1}ar^k = \sum_{k = 1}^{n}ar^{k-1} = a \cdot \frac{1 - r^n}{1 - r}}$ avec $n \in \N \text{ et } a, r \in \R$
\end{enumerate}
\begin{proof}
	\ref{1.3.1_sum1} On pose S = $\displaystyle{\sum_{k = 1}^{n} k}$ avec $n \in \N$ \\
	On a donc :
	\begin{align*}
		S = & 1 + 2 + \cdots + (n - 1) + n \\
		& n + (n - 1) + \cdots + 2 + 1 = S 
	\end{align*}
	En additionnant les termes du "dessus" et du "dessous" on obtient :
	\begin{align*}
		&2S = n \cdot (n+1) \\
		&S = \frac{n(n+1)}{2}
	\end{align*}
\end{proof}

\begin{proof}
	\ref{1.3.1_sum2} On pose S = $\displaystyle{\sum_{k = 1}^{n}\left(a + \left(k - 1\right)d\right)}$ avec $n \in \N,\ a, d \in \R$
	\begin{align*}
		S = \sum_{k = 1}^{n} \left(a + \left(k - 1\right)d\right) &= \sum_{k=1}^{n} (a - d + dk)\\
		&= \sum_{k=1}^{n} (a - d) + \sum_{k = 1}^{n} dk \\
		&= \sum_{k=1}^{n} (a - d) + d\sum_{k=1}^{n}k \\
		&= n(a - d) + d \frac{n(n + 1)}{2} \\
		&= n\left(\left(a - d\right) + \frac{d(n+1)}{2} \right) \\
		&= \frac{1}{2}n \left(2\left(a-d)\right) + d(n+1) \right) \\
		&= \frac{1}{2}n (2a - 2d + nd + d) \\
		&= \frac{1}{2}n (2a -d + nd) \\
		&= \frac{1}{2}n (2a + (n - 1)d)
	\end{align*}
\end{proof}

\begin{proof}
	\ref{1.3.1_sum3} On pose S = $\displaystyle{\sum_{k = 0}^{n-1}ar^k}$ avec $n \in \N,\ a, r \in \R$
	\begin{align*}
		&\begin{aligned}
			S &= a + ar + \cdots + ar^{n-1} \\
			rS &= ar + ar^2 + \cdots + ar^n 
		\end{aligned} 
		\\
		\\
		&\begin{aligned}
			S - rS = (a + &ar + \cdots + ar^{n-1}) \\
			- (&ar + \cdots + ar^{n-1} + ar^n)
		\end{aligned}
		\\
		\\
		&\begin{aligned}
			(1 - r)S &= a - ar^n \\
			S &= a \cdot \frac{1 - r^n}{1 - r}
		\end{aligned}
	\end{align*}
\end{proof}

\subsection{Sommes téléscopiques}
\begin{propositionbox}
    \begin{proposition}[Somme téléscopique]
	$\forall m, n \in \N \textnormal{ avec }m \leq n,\ a_{k}  \in \R \textnormal{ avec } m \leq k \leq n$
	\begin{align*}
        \boxshadow{\sum_{k = m}^{n} (a_k - a_{k-1}) = a_n - a_{m - 1} }
	\end{align*}
\end{proposition}
\end{propositionbox}

\begin{proof}
	$\forall m, n \in \N \textnormal{ avec }m \leq n,\ a_{k}  \in \R \textnormal{ avec } m \leq k \leq n$
	\begin{align*}
		&\begin{aligned}
			\sum_{k = m}^{n}(a_k + a_{k - 1}) = &(\underline{a_m} - a_{m-1}) \\
			+&(\underline{a_{m+1}} - \underline{a_m}) \\
			+&(a_{m+2} - \underline{a_{m+1}}) \\
			& \qquad \vdots \\
			+& (\underline{a_{n-1}} - \underline{a_{n-2}}) \\	
			+& (a_n - \underline{a_{n - 1}}) \\
		\end{aligned}
		\\
		&\begin{aligned}
			\sum_{k = m}^{n}(a_k + a_{k - 1}) = a_n - a_{m - 1}
		\end{aligned}
	\end{align*}
\end{proof}

\section{Puissances}

\begin{propositionbox}
    \begin{definition}[Puissance d'un réel]
	$\forall a \in \R,\ \forall n \in \N$
	\begin{align*}
        \boxshadow{a^n = \underbrace{a \times a \times \cdots \times a}_{n fois}}	
	\end{align*}
\end{definition}
\end{propositionbox}

\begin{propositionbox}
    \begin{proposition}[Propriétés des puissances]
	$\forall a \in \R,\ \forall m, n \in \N$
	\begin{enumerate}
        \item \label{1.4_power1} $\boxshadow{a^m \times a^n = a^{m + n}}$
        \item \label{1.4_power2} $\boxshadow{(a^m)^n = a^{mn}}$
        \item \label{1.4_power3} $\boxshadow{\displaystyle{\frac{a^n}{a^m}} = a^{n - m}, a \neq 0}$
        \item \label{1.4_power4} $\boxshadow{a^{-m} = \displaystyle{\frac{1}{a^m}}, a \neq 0}$
        \item \label{1.4_power5} $\boxshadow{a^0 = 1}$
	\end{enumerate}
\end{proposition}
\end{propositionbox}
\begin{proof}
	\ref{1.4_power1} $\forall a \in \R,\ \forall n, m \in \N$
	\begin{align*}
		a^m \times a^n &= \underbrace{(a \times a \times \cdots \times a)}_{m fois} \times \underbrace{(a \times a \times \cdots \times a)}_{n fois} \\
		&= \underbrace{a \times a \times \cdots \times a}_{m + n fois}
	\end{align*}
\end{proof}

\begin{proof}
	\ref{1.4_power2} $\forall a \in \R,\ \forall n, m \in \N$
	\begin{align*}
		(a^m)^n = \underbrace{\underbrace{(a \times a \times \cdots \times a)}_{m fois} \times \underbrace{(a \times a \times \cdots \times a)}_{m fois} \times \cdots \times \underbrace{(a \times a \times \cdots \times a)}_{m fois}}_{n fois} = \underbrace{a \times a \times \cdots \times a}_{m \times n fois}
	\end{align*}
\end{proof}

\begin{proof}
	\ref{1.4_power3} $\forall a \in \R^*,\ \forall n, m \in \N$
	\begin{align*}
		\frac{a^n}{a^m} = \frac{\overbrace{a \times a \cdots \times a}^{n fois}}{\underbrace{a \times a \cdots \times a}_{m fois}} = \underbrace{a \times a \times \cdots \times a}_{n - m fois}
	\end{align*}
\end{proof}

\begin{proof}
	\ref{1.4_power4} $\forall a \in \R^*,\ \forall m \in \N$
	\begin{align*}
		a^{-m} &= a^{0 - m}  \\
		&= \frac{a^0}{a^m} \\
		&= \frac{1}{a^m}
	\end{align*}
\end{proof}

\begin{proof}
	\ref{1.4_power5} $\forall a \in \R$
	\begin{align*}
		&a^1 = a \\
		&a^0 = \frac{a}{a} = 1
	\end{align*}
\end{proof}
