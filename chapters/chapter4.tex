\chapter{Nombres complexes}
\begin{align*}
(\N, +, \times) \subset (\Z, +, \times) \subset (\Q, +, \times) \subset (\R, +, \times) \subset (\C, +, \times)
\end{align*}
\par L'ensemble des nombres complexes est adapté à la résolution des équations algébriques.

\section{Vision algébrique des nombres complexes}


    \begin{definition}[Forme algébrique des nombres complexes]
\begin{align*}
    \C = \{a + ib \ | \ (a, b) \in \R^2 \}, \text{ avec } i = \sqrt{-1}
\end{align*}
\end{definition}


    \begin{proposition}[Opérations sur les nombres complexes]~
\begin{enumerate}
\item Somme : Soient $z = a + ib \in \C, \omega = c + id \in \C, (a, b, c, d) \in \R^4$
\begin{align*}
    z + \omega = a + c + i (b + d)
\end{align*}

\begin{enumerate}
    \item Associativité : $(z_1 + z_2) + z_3 = z_1 + (z_2 + z_3), (z_1, z_2, z_3) \in \C^3$
    \item Elément neutre : $0 = 0 + i0 \implies z + 0 = 0 + z = z,\ z \in \C$
\item Symétrique : $\forall z \in \C, \exists z', z + z' = z' + z = 0, z' = -z$
\begin{align*}
    z = a + ib \implies -z = -a + i(-b)
\end{align*}
\item Commutativité : $z + \omega = \omega + z$
\end{enumerate}

\item Produit : Soient $z = a + ib \in \C, \omega = c + id \in \C, (a, b, c, d) \in \R^4$
\begin{align*}
    z \cdot \omega = (ac - bd) + i(ad + bc)
\end{align*}

\begin{enumerate}
    \item Associativité :
    \begin{align*}
        (z_1 \cdot z_2) \cdot z_3 = z_1 \cdot (z_2 \cdot z_3), \forall (z_1, z_2, z_3) \in \C^3 
    \end{align*}
    \item Elément neutre :
        \begin{align*}
            1 = 1 + i0 \implies z \times 1 = 1 \times z = z \\
            \forall z \in \C \backslash \{0\}, \exists z' \in \C, z \cdot z' = z' \cdot z =1 
        \end{align*}
    \item Commutativité :
        \begin{align*}
            z \cdot \omega = \omega \cdot z, \forall (z, \omega) \in \C^2 
        \end{align*}
    \item Distributivité :
        \begin{align*}
        &(z_1 + z_2) \cdot \omega = z_1 \cdot \omega + z_2 \cdot \omega \\
        &z \cdot (\omega_1 + \omega_2) = z \cdot \omega_1 + z \cdot \omega_2 \\
        &\forall (z, z_1, z_2, \omega, \omega_1, \omega_1) \in \C^6
        \end{align*}
\end{enumerate}

\end{enumerate}
\end{proposition}

\begin{proof}{Produit}
\begin{align*}
z \cdot \omega &= (a + ib) \cdot (c + id) \\
          &\mlqq=\mrqq a \cdot (c + id) + ib \cdot (c + id) \\
          &\mlqq=\mrqq a \cdot c + a \cdot id + ib \cdot c + ib \cdot id \\
          &\mlqq=\mrqq ac + i(ad) + i(bc) + i^2 bd \\
          &\mlqq=\mrqq ac - bd + i(ad + bc)
\end{align*}
\end{proof}


\begin{remarque}
    $(\C, +, \times)$ est un corps commutatif 
\end{remarque}



    \begin{definition}[Conjugué d'un nombre complexe]
    Soit $z = a + ib$ un nombre complexe, le nombre $\overline{z} = a - ib$ est dit le conjugué de z.
\end{definition}


\begin{proposition}{}
    Soient $z = a + ib, z'= a - ib, (a, b) \in \R^2, z \in \C$
    \begin{align*}
        z \cdot z' = a^2 + b^2 
    \end{align*}
\end{proposition}
\begin{proof}
        \begin{align*}
            z \cdot z' &= (a + ib)(a - ib) \\
                        &= a^2 -iab + iab -i^2b^2 \\
                        &= a^2 + b^2
        \end{align*}
    \end{proof}



    \begin{definition}[Module d'un nombre complexe]
    Soit $z = a + ib$ un nombre complexe, on définit son module comme étant :
    \begin{align*}
        |z| = \sqrt{a^2 + b^2} 
    \end{align*}
\end{definition}


    \begin{proposition}[Propriétés des modules]
    Soient $z = a + ib$ et $z = a' + ib'$ des nombres complexes, on a les propriétés suivantes sur les modules :
    \begin{itemize}
        \item $|z \cdot z'| = |z| \cdot |z'|$
        \item $\left| \frac{z}{z'} \right| = \frac{|z|}{|z'|}$
        \item $|z + z'| \leq |z| + |z'|$
        \item $|z|^2 = z \cdot \overline{z} = a^2 + b^2$
        \item $|z| \geq 0$
        \item $|z| = 0 \iff z = 0$
        \item $|z| = |\overline{z}| = |-z| = |-\overline{z}|$
    \end{itemize}
\end{proposition}


    \begin{definition}[Partie réelle et partie imaginaire]
    Soit $z = a + ib \in \C, (a, b) \in \R^2$
    \begin{align*}
        &\Re(z) = Re(z) = a \text{ (Partie réelle)} \\
        &\Im(z) = Im(z) = b \text{ (Partie imaginaire)}
    \end{align*}
\end{definition}


\begin{proposition}~
    \begin{itemize}
        \item $z + \overline{z} = (a + ib) + (a - ib) = 2a \implies \Re(z) = \dfrac{z + \overline{z}}{2}$
        \item $z - \overline{z} = (a + ib) - (a - ib) = 2ib \implies \Im(z) = \dfrac{z - \overline{z}}{2i}$
    \end{itemize}
\end{proposition}
\section{Vision géométrique des nombres complexes}

    \begin{definition}[Argument d'un nombre complexe]
    Soit $z$ un nombre complexe, l'argument de $z$, noté $\arg{(z)}$ représente l'angle entre la droite des réels et celle issue de l'origine et passant par $z$.
\end{definition}


    \begin{proposition}[Propriétés des arguments]
    Soient $z, z_1, z_2 \in \C^3, n \in \N$
    \begin{itemize}
        \item $\arg{(z_1 \cdot z_2)} = \arg{z_1} + \arg{z_2}$
        \item $\arg{z^n} = n\arg{z}$
        \item $\arg{\frac{z_1}{z_2}} = \arg{z_1} - \arg{z_2}$
        \item $\arg{\frac{1}{z}} = -\arg{z}$
    \end{itemize}
\end{proposition}


    \begin{definition}[Forme trigonométrique d'un nombre complexe]
    Soit $z$ un nombre complexe, on peut l'écrire sous sa forme trigonométrique ainsi :
    \begin{align*}
        z = r (\cos{(\theta)} + i \sin{(\theta)}) 
    \end{align*}
    Avec :
    \begin{itemize}
        \item $r = |z|$
        \item $\theta = \arg{(z)}$
    \end{itemize}
\end{definition}


\begin{proposition}{}
    Soient $z_1 = r_1(\cos{(\theta_1)} + i \sin{(\theta_1)})$ et $z_2 = r_2(\cos{(\theta_2)} + i\sin{(\theta_2)})$, deux nombres complexes. Nous avons la propriété suivante :
    \begin{align*}
        z_1 z_2 = r_1r_2(\cos{(\theta_1 + \theta_2)} + i\sin{(\theta_1 + \theta_2)})
    \end{align*}

\end{proposition}
\begin{proof}
        \begin{align*}
            z_1 z_2 &= (r_1(\cos{(\theta_1)} + i \sin{(\theta_1)}) (r_2(\cos{(\theta_2)}+i \sin{(\theta_2)}) \\
                    &=(r_1\cos{\theta_1} + ir_1 \sin{\theta_1}) (r_2\cos{\theta_2} + ir_2 \sin{\theta_2}) \\
                    &= (r_1\cos{\theta_1} \cdot r_2\cos{\theta_2}) + (r_1\cos{\theta_1} \cdot ir_2\sin{\theta_2}) + (ir_1\sin{\theta_1} \cdot r_2\cos{\theta_2}) + (ir_1\cos{\theta_1} + ir_2\sin{\theta_2})  \\
                    &= (r_1\cos{\theta_1})(r_2\cos{\theta_2}) - (r_1\sin{\theta_1})(r_2\sin{\theta_2}) + i((r_1\cos{\theta_1})(r_2\sin{\theta_2}) + (r_1\sin{\theta_1})(r_2\cos{\theta_2})) \\
                    &= r_1r_2((\cos{\theta_1} \cos{\theta_2} - \sin{\theta_1} \sin{\theta_2}) + i(\sin{\theta_1}\cos{\theta_2} + \cos{\theta_1}\sin{\theta_2})) \\
                    &= r_1r_2(\cos{(\theta_1 + \theta_2)} + i\sin{(\theta_1 + \theta_2)})
        \end{align*}
    \end{proof}

    \begin{proposition}[Formule de Moivre]
    \begin{align*}
        (\cos{\theta} + i \sin{\theta})^n = \cos{(n\theta)} + i\sin{(n\theta)} 
    \end{align*}
\end{proposition}


    \begin{definition}[Forme exponentielle d'un nombre complexe]
    On peut écrire un nombre complexe sous une forme exponentielle :
    \begin{align*}
        z = r(\cos{\theta} + i\sin{\theta}) = re^{i\theta}
    \end{align*}
\end{definition}


    \begin{proposition}[Identité d'Euler]
    \begin{align*}
        e^{i \pi} = -1 
    \end{align*}
\end{proposition}

    \begin{proposition}[Formules d'Euler]
    \begin{align*}
        \cos{\theta} &= \frac{e^{i\theta} + e^{-i\theta}}{2} & \sin{\theta} &= \frac{e^{i\theta} - e^{-i\theta}}{2i}
    \end{align*}
\end{proposition}
\begin{proof}
        \begin{align*}
            e^{i\theta} + e^{-i\theta} &= (\cos{\theta} + i\sin{\theta}) + (\cos{\theta} - i\sin{\theta}) \\
                                       &= 2\cos{\theta} \\
        \frac{e^{i\theta} + e^{-i\theta}}{2} &= \cos{\theta}
        \end{align*}
        
        \begin{align*}
            e^{i\theta} - e^{-i\theta} &= (\cos{\theta} + i\sin{\theta}) - (\cos{\theta} - i\sin{\theta}) \\
                                       &= 2i\sin{\theta} \\
            \frac{e^{i\theta} - e^{-i\theta}}{2i} &= \sin{\theta} 
        \end{align*}
\end{proof}


\begin{remarque}[Passer de la forme algébrique à la forme trigonométrique]~ 
    \\
    Soit $z = a + ib, (a, b) \in \R^2$ un nombre complexe sous sa forme algébrique, on peut passer sous la forme trigonométrique ainsi :
    \begin{align*}
        \cos{\theta} &= \frac{a}{|z|} & \sin{\theta} &= \frac{b}{|z|}
    \end{align*}
    \end{remarque}



\begin{exemple}
        $z = 1 + i$
        \\
        On a : $|z| = \sqrt{1^2 + 1^2}$
        \\
        On a donc :
        \begin{align*}
            \cos{\theta} &= \frac{1}{\sqrt{2}} & \sin{\theta} &= \frac{1}{\sqrt{2}} \\
                         &= \frac{\sqrt{2}}{2} &              &= \frac{\sqrt{2}}{2}
        \end{align*}
        On en déduit donc que $\theta = \frac{\pi}{4}$. \\
        Ainsi $z = \sqrt{2}\left(\cos{\frac{\pi}{4}} + i\sin{\frac{\pi}{4}}\right)$
    \end{exemple}



    \begin{definition}[Racine n-ième d'un nombre complexe]
    Soit $z \in \C$. On appelle racine n-ième du nombre complexe $z$ tout nombre complexe $\omega \in \C$ vérifiant :
    \begin{align*}
        \omega^n = z 
    \end{align*}
\end{definition}

\begin{proposition}
Un complexe non nul $z = \rho e^{i\theta}$ ($\rho = |z|$) admet n racines n-ièmes données par :
    \begin{align*}
        \omega = \rho^{\frac{1}{n}} e^{i\left(\frac{\theta}{n} + \frac{2k\pi}{n} \right)} 
    \end{align*}
\end{proposition}

\begin{proof}
On cherche à résoudre
\begin{align*}
\omega^n = z, \ (n \in \N)
\end{align*}
Posons
\begin{align*}
\begin{cases}
\omega = |\omega|e^{i \theta_1} \\
z = |z| e^{i \theta_2}
\end{cases} 
\iff 
\begin{cases}
\omega^n = |\omega^n|e^{i n \theta_1} \\
z = |z| e^{i \theta_2}
\end{cases}
\end{align*}
\\
Par identification :
\begin{align*}
\begin{cases}
|\omega|^n = |z| \\
n \theta_1 = \theta_2 + 2k \pi,\ (k \in \Z)
\end{cases}
\iff 
\begin{cases}
|\omega| = |z|^{\frac{1}{n}} \\
\theta_1 = \frac{\theta_2}{n} + \frac{2 k \pi}{n},\ (k \in \Z)
\end{cases}
\end{align*}
En posant :
\begin{align*}
\begin{cases}
\rho = |z| \\
\theta_2 = \theta
\end{cases}
\end{align*}
on obtient :
\begin{align*}
\omega = \rho^{\frac{1}{n}} e^{i \left( \frac{\theta}{n} + \frac{2k\pi}{n} \right)},\ (k \in \Z)
\end{align*}
\end{proof}

\begin{definition}[Racine n-ième de l'unité]
    On appelle racine n-ième de l'unité, une racine n-ième de 1, on notera $\mathbb{U}_n$ l'ensemble des racines n-ièmes de l'unité :
    \begin{align*}
        \mathbb{U}_n = \{z \in \C | z^n = 1 \} 
    \end{align*}
\end{definition}

\begin{proposition}
Les racines n-ièmes de l'unité sont de la forme :
    \begin{align*}
        \omega_k = e^{\frac{2ik\pi}{n}}, k \in \llbracket 0, n - 1 \rrbracket 
    \end{align*}
\end{proposition}

\begin{proof}
On cherche à résoudre 
\begin{align*}
z^n = 1,\ n \in \N
\end{align*}
\\
Posons $z = |z|e^{i \theta}$. On obtient donc en elevant à la puissance $n$
\begin{align*}
z^n = |z^n|e^{i n \theta} &= 1 \\
\iff |z|^ne^{i n \theta} &= e^{i 0} 
\end{align*}
Par identification 
\begin{align*}
\begin{cases}
|z| = 1 \\
n \theta = 0 + 2 k \pi,\ (k \in \Z)
\end{cases}
\end{align*}
On obtient alors 
\begin{align*}
\theta = \frac{2k \pi}{n},\ (k \in \Z)
\end{align*}
Finalement on obtient bien 
\begin{align*}
z = e^{i \frac{2 k \pi}{n}},\ (k \in \Z)
\end{align*}
que l'on peut également écrire 
\begin{align*}
z = e^{i \frac{2 k \pi}{n}},\ (k \in \llbracket 0, n - 1 \rrbracket )
\end{align*}
car il y a un cycle.
\end{proof}

\section{Géométrie des nombres complexes}
\begin{itemize}
    \item $z \mapsto z + a, (a \in \C)$ : translation de vecteur $\vec{u}$ d'affixe $a$
    \item $z \mapsto az, (a \in \R^*)$ : homothétie de rapport $a$
    \item $z \mapsto e^{i\theta}z, (\theta \in \R)$ : rotation d'angle $\theta$ et de centre $0$
    \item $z \mapsto \overline{z}$ : réflexion par rapport à l'axe des réels
    \item $z \mapsto a + e^{i\theta} (z - a)$ : rotation d'angle $\theta$ de centre $a$
    \item $z \mapsto e^{2i\theta} \cdot \overline{z}$ : réflexion par rapport à la droite formant un angle $\theta$ avec l'axe des réels.
\end{itemize}

\subsection{Equation d'une droite}
\begin{itemize}
    \item L'axe des réels : $\overline{z} = z$
    \item Un axe formant un angle $\theta$ avec l'axe des réels : $\overline{e^{-i\theta}z}= e^{-i\theta}z$
    \item L'asymptote verticale de partie réelle $a$ : $z + \overline{z} = 2a$
\end{itemize}


\begin{exemple}
    $z \mapsto \frac{1}{z}$ 
    \\
    On pose : $\omega = \frac{1}{z}$
    \\
    On a donc : $z = \frac{1}{\omega}$
    \\
    $z + \overline{z} = 2 \implies \frac{1}{\omega} + \overline{\frac{1}{\omega}} = 2 \implies \frac{1}{\omega} + \frac{1}{\overline{\omega}} = 2$ \\
    $\omega\overline{\omega} \left( \frac{1}{\omega} + \frac{1}{\overline{\omega}} \right) = 2\omega\overline{\omega}$
    \\
    On a donc : $\overline{\omega} + \omega = 2\omega\overline{\omega} \implies 2\omega\overline{\omega} - \omega - \overline{\omega} = 0$ 
    \\
    C'est à dire : $\omega\overline{\omega} - \frac{1}{2}\omega - \frac{1}{2}\overline{\omega} = \left( \omega - \frac{1}{2} \right) \left( \overline{\omega} - \frac{1}{2} \right) - \frac{1}{4} = 0$
    \\
    Ce qui équivaut à $\left| \omega - \frac{1}{2} \right|^2 = \left( \frac{1}{2} \right)^2 \iff \left|\omega - \frac{1}{2}\right| = \left(\frac{1}{2}\right)$
\end{exemple}



\begin{exemple}
    $P = \left\{ z \in \C, \Im{(z)} > 0\right\}$ : le demi-plan de Poincaré    
    \\
    Déterminer l'image de $P$ par la transformation $z \mapsto \frac{z - i}{z + i}$
    \\
    \begin{enumerate}
        \item $\omega = \frac{z - i}{z + i}$, exprimer $z$ en fonction de $\omega$.
            \begin{align*}
                &\omega = \frac{z - i}{z + i} \\
                \iff &\omega (z + i) = z - i \\
                \iff &\omega (z + i) + i = z \\
                \iff &\omega z + \omega i + i = z \\
                \iff &\omega z - z = -\omega i - i \\
                \iff &z(\omega + 1) = -\omega i - i \\
                \iff &z = \frac{-\omega i - i}{\omega + 1} \\
                \iff &z = \frac{-i(\omega + 1)}{\omega + 1} 
            \end{align*}
        \item $z \in P \iff \Im{(z)} > 0$ 
            \\
            $z = x + iy$, $\overline{z} = x - iy$, on a : $z - \overline{z} = 2iy$ \\
            Si on a $\Im{(z)} = y > 0 \iff \frac{1}{2i}(z - \overline{z}) > 0$
            \\
            A la fin on obtient : $\omega\overline{\omega} < 1 \implies |\omega| < 1$
    \end{enumerate}
\end{exemple}

