\chapter{Logique}
\begin{definition}[Assertion]
\par Une \textbf{assertion} est une affirmation mathématique qui peut être vraie pou fausse.
\end{definition}

\begin{definition}[Prédicat]
\par Un \textbf{prédicat} est une "assertion" dépendant d'une ou plusieurs variables.
\end{definition}

\begin{exemple}~
\begin{itemize}
\item "Tous les entiers sont des nombres rationnels" est une assertion.
\item "L'entier n est pair" est un prédicat.
\item "Le réel $x$ est le carré d'un nombre réel" est un prédicat.
\end{itemize}
\end{exemple}

\section{Opérations sur les prédicats}
\begin{table}[h!]
\centering
\begin{tabular}{|c|c|c|c|c|c|}
\hline
P & Q & P et Q & P ou Q & non(P) & P $\implies$ Q \\
\hline
V & V & V & V & F & V \\
\hline
V & F & F & V & F & F \\
\hline
F & V & F & V & V & V \\
\hline 
F & F & F & F & V & V \\
\hline
\end{tabular}
\end{table}

\subsection{Négation}
\begin{enumerate}
\item $P \implies Q$ est équivalent à non(P) ou Q
\item non(P ou Q) est équivalent à non(P) et non(Q)
\item non(P et Q) est équivalent à non(P) ou non(Q)
\end{enumerate}

\begin{remarque}~
\begin{enumerate}
\item Pour contredire "tous les éléments de E ont une propriété P", il suffit de trouver un contre-exemple

\begin{align*}
    non(\forall x \in E, P(x)) \iff \exists x \in E, non(P(x))
\end{align*}   

\item Pour contredire "il existe un élément de E vérifiant une propriété P", il faut montrer que tous les éléments de E ne vérifient pas la propriété P.

\begin{align*}
non(\exists x \in E, P(x)) \iff \forall x \in E, non(P(x))
\end{align*}

\item Une affirmation de type :
\begin{align*}
\exists ! x \in E, P(x) \iff 
\begin{cases}
& \exists x \in E, P(x) \\
& \text{Si } P(x) \text{ et } P(y) \text{ sont vrais, alors } x = y
\end{cases}
\end{align*}
\end{enumerate}
\end{remarque}

\begin{remarque}~
\\
    $\{ (a_n) \}_{n \in \N} \subset \R, \displaystyle \lim_{n \to +\infty} a_n = \alpha \in \R$ \\
A. Cauchy : 
\begin{align*}
    \forall \varepsilon > 0, \exists N, \left|a_n - \alpha \right| < \varepsilon, \forall n \geq N
\end{align*}
\end{remarque}


