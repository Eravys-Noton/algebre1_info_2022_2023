\chapter{Nombres complexes}
\begin{align*}
(\N, +, \times) \subset (\Z, +, \times) \subset (\Q, +, \times) \subset (\R, +, \times) \subset (\C, +, \times)
\end{align*}
\par L'ensemble des nombres complexes est adapté à la résolution des équations algébriques.

\section{Vision algébrique des nombres complexes}

\begin{definition}[Forme algébrique des nombres complexes]
\begin{align*}
\C = \{a + ib \ | \ (a, b) \in \R^2 \}, \text{ avec } i = \sqrt{-1}
\end{align*}
\end{definition}

\begin{proposition}[Opérations sur les nombres complexes]~
\begin{enumerate}
\item Somme : Soient $z = a + ib \in \C, w = c + id \in \C, (a, b, c, d) \in \R^4$
\begin{align*}
z + w = a + c + i (b + d)
\end{align*}

\begin{enumerate}
\item Associativité : $(z_1 + z_2) + z_3 = z_1 + (z_2 + z_3), (z_1, z_2, z_3) \in \C^3$
\item Elément neutre : $0 = 0 + i0 \implies z + 0 = 0 + z = 0, z \in \C$
\item Symétrique : $\forall z \in \C, \exists z', z + z' = z' + z = 0, z' = -z$
\begin{align*}
z = a + ib \implies -z = -a + i(-b)
\end{align*}
\item Commutativité : $z + w = w + z$
\end{enumerate}

\item Produit : Soient $z = a + ib \in \C, w = c + id \in \C, (a, b, c, d) \in \R^4$
\begin{align*}
z \cdot w = (ac - bd) + i(ad + bc)
\end{align*}

\begin{proof}
\begin{align*}
z \cdot w &= (a + ib) \cdot (c + id) \\
          &\mlqq=\mrqq a \cdot (c + id) + ib \cdot (c + id) \\
          &\mlqq=\mrqq a \cdot c + a \cdot id + ib \cdot c + ib \cdot id \\
          &\mlqq=\mrqq ac + i(ad) + i(bc) + i^2 bd \\
          &\mlqq=\mrqq ac - bd + i(ad + bc)
\end{align*}
\end{proof}

\begin{enumerate}
    \item Associativité :
    \begin{align*}
    (z_1 \cdot z_2) \cdot z_3 = z_1 \cdot (z_2 \cdot z_3), \forall (z_1, z_2, z_3) \in \C^3
    \end{align*}
    \item Elément neutre :
        \begin{align*}
        1 = 1 + i0 \implies z \times 1 = 1 \times z = z \\
        \forall z \in \C \backslash \{0\}, \exists z' \in \C, z \cdot z' = z' \cdot z =1
        \end{align*}
    \item Commutativité :
        \begin{align*}
        z \cdot w = w \cdot z, \forall (z, w) \in \C^2
        \end{align*}
    \item Distributivité :
        \begin{align*}
        &(z_1 + z_2) \cdot w = z_1 \cdot w + z_2 \cdot w \\
        &z \cdot (w_1 + w_2) = z \cdot w_1 + z \cdot w_2 \\
        &\forall (z, z_1, z_2, w, w_1, w_1) \in \C^6
        \end{align*}
\end{enumerate}

\end{enumerate}
\end{proposition}

\begin{remarque}
    $(\C, +, \times)$ est un corps commutatif 
\end{remarque}

\begin{definition}[Conjugué d'un nombre complexe]
    Soit $z = a + ib$ un nombre complexe, le nombre $\overline{z} = a - ib$ est dit le conjugué de z.
\end{definition}

\begin{proposition}
    Soient $z = a + ib, z'= a - ib, (a, b) \in \R^2, z \in \C$
    \begin{align*}
        z \cdot z' = a^2 + b^2
    \end{align*}
    \begin{proof}
        \begin{align*}
            z \cdot z' &= (a + ib)(a - ib) \\
                        &= a^2 -iab + iab -i^2b^2 \\
                        &= a^2 + b^2
        \end{align*}
    \end{proof}
\end{proposition}

\begin{definition}[Module d'un nombre complexe]
    Soit $z = a + ib$ un nombre complexe, on définit son module comme étant :
    \begin{align*}
        |z| = \sqrt{a^2 + b^2}
    \end{align*}
\end{definition}

\begin{proposition}[Propriétés des modules]
    Soient $z = a + ib$ et $z = a' + ib'$ des nombres complexes, on a les propriétés suivantes sur les modules :
    \begin{itemize}
        \item $|z \cdot z'| = |z| \cdot |z'|$
        \item $\left| \frac{z}{z'} \right| = \frac{|z|}{|z'|}$
        \item $|z + z'| \leq |z| + |z'|$
        \item $|z|^2 = z \cdot \overline{z} = a^2 + b^2$
        \item $|z| \geq 0$
        \item $|z| = 0 \iff z = 0$
        \item $|z| = |\overline{z}| = |-z| = |-\overline{z}|$
    \end{itemize}
\end{proposition}

\begin{definition}[Partie réelle et partie imaginaire]
    Soit $z = a + ib \in \C, (a, b) \in \R^2$
    \begin{align*}
        &\Re(z) = Re(z) = a \text{ (Partie réelle)} \\
        &\Im(z) = Im(z) = b \text{ (Partie imaginaire)}
    \end{align*}
\end{definition}

\begin{proposition}[Propriétés]~
    \begin{itemize}
        \item $z + \overline{z} = (a + ib) + (a - ib) = 2a \implies \Re(z) = \dfrac{z + \overline{z}}{2}$
        \item $z - \overline{z} = (a + ib) - (a - ib) = 2ib \implies \Im(z) = \dfrac{z - \overline{z}}{2i}$
    \end{itemize}
\end{proposition}

\section{Vision géométrique des nombres complexes}

\begin{definition}[Argument d'un nombre complexe]
    Soit $z$ un nombre complexe, l'argument de $z$, noté $\arg{(z)}$ représente l'angle entre la droite des réels et celle issue de l'origine et passant par $z$.
\end{definition}

\begin{proposition}[Propriétés des arguments]~
    Soient $z, z_1, z_2 \in \C^3, n \in \N$
    \begin{itemize}
        \item $\arg{(z_1 \cdot z_2)} = \arg{z_1} + \arg{z_2}$
        \item $\arg{z^n} = n\arg{z}$
        \item $\arg{\frac{z_1}{z_2}} = \arg{z_1} - \arg{z_2}$
        \item $\arg{\frac{1}{z}} = -\arg{z}$
    \end{itemize}
\end{proposition}

\begin{definition}[Forme trigonométrique d'un nombre complexe]
    Soit $z$ un nombre complexe, on peut l'écrire sous sa forme trigonométrique ainsi :
    \begin{align*}
        z = r (\cos{(\theta)} + i \sin{(\theta)})
    \end{align*}
    Avec :
    \begin{itemize}
        \item $r = |z|$
        \item $\theta = \arg{(z)}$
    \end{itemize}
\end{definition}

\begin{proposition}
    Soient $z_1 = r_1(\cos{(\theta_1)} + i \sin{(\theta_1)})$ et $z_2 = r_2(\cos{(\theta_2)} + i\sin{(\theta_2)})$, deux nombres complexes. Nous avons la propriété suivante :
    \begin{align*}
        z_1 z_2 = r_1r_2(\cos{(\theta_1 + \theta_2)} + i\sin{(\theta_1 + \theta_2)})
    \end{align*}

    \begin{proof}
        \begin{align*}
            z_1 z_2 &= (r_1(\cos{(\theta_1)} + i \sin{(\theta_1)}) (r_2(\cos{(\theta_2)}+i \sin{(\theta_2)}) \\
                    &=(r_1\cos{\theta_1} + ir_1 \sin{\theta_1}) (r_2\cos{\theta_2} + ir_2 \sin{\theta_2}) \\
                    &= (r_1\cos{\theta_1} \cdot r_2\cos{\theta_2}) + (r_1\cos{\theta_1} \cdot ir_2\sin{\theta_2}) + (ir_1\sin{\theta_1} \cdot r_2\cos{\theta_2}) + (ir_1\cos{\theta_1} + ir_2\sin{\theta_2})  \\
                    &= (r_1\cos{\theta_1})(r_2\cos{\theta_2}) - (r_1\sin{\theta_1})(r_2\sin{\theta_2}) + i((r_1\cos{\theta_1})(r_2\sin{\theta_2}) + (r_1\sin{\theta_1})(r_2\cos{\theta_2})) \\
                    &= r_1r_2((\cos{\theta_1} \cos{\theta_2} - \sin{\theta_1} \sin{\theta_2}) + i(\sin{\theta_1}\cos{\theta_2} + \cos{\theta_1}\sin{\theta_2})) \\
                    &= r_1r_2(\cos{(\theta_1 + \theta_2)} + i\sin{(\theta_1 + \theta_2)})
        \end{align*}
    \end{proof}
\end{proposition}

\begin{proposition}[Formule de Moivre]
    \begin{align*}
        (\cos{\theta} + i \sin{\theta})^n = \cos{(n\theta)} + i\sin{(n\theta)}
    \end{align*}
\end{proposition}

\begin{definition}[Forme exponentielle d'un nombre complexe]
    On peut écrire un nombre complexe sous une forme exponentielle :
    \begin{align*}
        z = r(\cos{\theta} + i\sin{\theta}) = re^{i\theta} 
    \end{align*}
\end{definition}

\begin{proposition}[Identité d'Euler]
    \begin{align*}
        e^{i \pi} = -1
    \end{align*}
\end{proposition}

\begin{proposition}[Formules d'Euler]
    \begin{align*}
        \cos{\theta} &= \frac{e^{i\theta} + e^{-i\theta}}{2} & \sin{\theta} &= \frac{e^{i\theta} - e^{-i\theta}}{2i}
    \end{align*}
    \begin{proof}
        \begin{align*}
            e^{i\theta} + e^{-i\theta} &= (\cos{\theta} + i\sin{\theta}) + (\cos{\theta} - i\sin{\theta}) \\
                                       &= 2\cos{\theta} \\
        \frac{e^{i\theta} + e^{-i\theta}}{2} &= \cos{\theta}
        \end{align*}
        
        \begin{align*}
            e^{i\theta} - e^{-i\theta} &= (\cos{\theta} + i\sin{\theta}) - (\cos{\theta} - i\sin{\theta}) \\
                                       &= 2i\sin{\theta} \\
            \frac{e^{i\theta} - e^{-i\theta}}{2i} &= \sin{\theta} 
        \end{align*}
    \end{proof}
\end{proposition}

\begin{remarque}[Passer de la forme algébrique à la forme trigonométrique]~ 
    \\
    Soit $z = a + ib, (a, b) \in \R^2$ un nombre complexe sous sa forme algébrique, on peut passer sous la forme trigonométrique ainsi :
    \begin{align*}
        \cos{\theta} &= \frac{a}{|z|} & \sin{\theta} &= \frac{b}{|z|}
    \end{align*}
    \begin{exemple}
        $z = 1 + i$
        \\
        On a : $|z| = \sqrt{1^2 + 1^2}$
        \\
        On a donc :
        \begin{align*}
            \cos{\theta} &= \frac{1}{\sqrt{2}} & \sin{\theta} &= \frac{1}{\sqrt{2}} \\
                         &= \frac{\sqrt{2}}{2} &              &= \frac{\sqrt{2}}{2}
        \end{align*}
        On en déduit donc que $\theta = \frac{\pi}{4}$. \\
        Ainsi $z = \sqrt{2}\left(\cos{\frac{\pi}{4}} + i\sin{\frac{\pi}{4}\right)$
    \end{exemple}
\end{remarque}
