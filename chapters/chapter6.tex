\chapter{Polynômes}

\begin{graybox}
    \begin{definition}[Polynôme]
        Un polynôme est un élément de l'ensemble 
        \begin{align*}
            \K[X] = \{ a_0 1 + a_1 X + a_2 X^2 + \cdots a_n X^n \mid a_i \in \K,\ n \in \N \}
        \end{align*}
    \end{definition}
\end{graybox}

\begin{remarque}
$\K$ désigne le corps étudié.
\end{remarque}
\begin{graybox}
    \begin{definition}[Fonction polynômiale]
        Une fonction polynomiale est une fonction $f$ définie par :
        \begin{align*}
            f \colon 
            \begin{cases}
                \K &\to \K \\
                x &\mapsto \sum_{i = 0}^{n} a_i x^i,\ a_i \in \K
            \end{cases}
        \end{align*}
    \end{definition}
\end{graybox}

Soient $P$ et $Q$ deux polynômes définis par 
                \begin{itemize}
                    \item $P(X) = a_0 1 + a_1 X + \cdots + a_n X^n$
                    \item $Q(X) = b_0 1 + b_1 X+ \cdots + b_n X^n$
                \end{itemize}

        \begin{enumerate}
            \item On a :
            \begin{align*}
                P(X) + Q(X) \colon = (a_0 + b_0) 1 + (a_1 + b_1) X + \cdots + (a_n + b_n) X^n
            \end{align*}
        \item Puis :
            \begin{align*}
                (\sum_{k = 0}^m a_k X^k) \cdot (\sum_{l = 0}^{n} b_l X^l) \colon = \sum_{k = 0}^m \sum_{l = 0}^n a_k b_l X^{k + l}
            \end{align*}
\end{enumerate}

\begin{graybox}
    \begin{proposition}
        Soient $P, Q, R$ trois polynômes.   
        \begin{enumerate}
            \item $(P + Q) + R = P + (Q + R)$
            \item $P + 0 = 0 + P = P$
            \item $P + (-P) = (-P) + P = 0$
            \item $(P \cdot Q) \cdot R = P \cdot (Q \cdot R)$
            \item $P \cdot 1 = 1 \cdot P = P$
            \item $P \cdot Q = Q \cdot P$ 
            \item $(P + Q) R = P \cdot R + Q \cdot R$
            \item $P \cdot (Q + R) = P \cdot Q + P \cdot R$
        \end{enumerate}
    \end{proposition}
\end{graybox}

\begin{graybox}
    \begin{definition}[Degré d'un polynôme]
        $P = a_0 + a_1 X + \cdots a_n X^n$ avec $(a_n \neq 0)$ $\in \K[X]$
        On définit $n$ comme étant le degré de $P$, on le note :
        \begin{align*}
            \mathrm{deg}(P) = n
        \end{align*}
        On peut aussi décrire le degré sous la forme d'une application :
        $\mathrm{deg} \colon \K[X] \to \N$
    \end{definition}
\end{graybox}

\begin{graybox}
    \begin{proposition}[Propriétés sur les degrés]~ 
        \begin{enumerate}
            \item Soit $\lambda \in \K$ on a : $\mathrm{deg}(\lambda) = 0$
            \item Soient $P$ et $Q$ deux polynômes on a : $\mathrm{deg}(P \cdot Q) = \mathrm{deg}(P) + \mathrm{deg}(Q)$
            \item Soient $P$ et $Q$ deux polynômes on a :
            $\mathrm{deg}(P + Q) = \mathrm{max}\left(\mathrm{deg}(P),\ \mathrm{deg}(Q)\right)$
        \end{enumerate}
    \end{proposition}
\end{graybox}

\begin{graybox}
    \begin{theoreme}
        $P_1, P_2$ deux polynômes non nuls.
        \begin{align*}
            \exists ! (Q, R) \in \K[X]^2 \text{ tel que } P_1 = P_2 Q + R 
        \end{align*}
        Vérifiant :
             $\mathrm{deg}(R) = -\infty$ ou $0 \leq \mathrm{deg}(R) < \mathrm{deg}(Q)$
    \end{theoreme}
\end{graybox}

\begin{exemple}
$P_1 = X^3 + 6X^2 + 11X + 6$ et $P_2 = X + 3$
\begin{align*}
X^3 + 6X^2 + 11X + 6 = (X+3)(X^2 + 13X + 2)  
\end{align*}
\end{exemple}

\begin{graybox}
\begin{theoreme}[Théorème fondamental de l'algèbre]
Soit $P(X)$ un polynôme à coefficients complexes de degré $n$. $P(X)$ admet $n$ racines complexes. 
\end{theoreme}
\end{graybox}

\begin{graybox}
\begin{proposition}
Soit $P(X)$ un polynôme, si $\alpha$ est une racine de $P(X)$ alors on a :
\begin{align*}
X - \alpha \mid P(X)
\end{align*}
\end{proposition}
\end{graybox}