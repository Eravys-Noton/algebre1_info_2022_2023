\chapter{Polynômes}

\section{Définitions}
\begin{graybox}
    \begin{definition}[Polynôme]
        Un polynôme est un élément de l'ensemble 
        \begin{align*}
            \K[X] = \{ a_0 1 + a_1 X + a_2 X^2 + \cdots a_n X^n \mid a_i \in \K,\ n \in \N \}
        \end{align*}
    \end{definition}
\end{graybox}

\begin{remarque}
$\K$ désigne le corps étudié.
\end{remarque}
\begin{graybox}
    \begin{definition}[Fonction polynômiale]
        Une fonction polynomiale est une fonction $f$ définie par :
        \begin{align*}
            f \colon 
            \begin{cases}
                \K &\to \K \\
                x &\mapsto \sum_{i = 0}^{n} a_i x^i,\ a_i \in \K
            \end{cases}
        \end{align*}
    \end{definition}
\end{graybox}

\section{Propriétés}
Soient $P$ et $Q$ deux polynômes définis par 
                \begin{itemize}
                    \item $P(X) = a_0 1 + a_1 X + \cdots + a_n X^n$
                    \item $Q(X) = b_0 1 + b_1 X+ \cdots + b_n X^n$
                \end{itemize}

        \begin{enumerate}
            \item On a :
            \begin{align*}
                P(X) + Q(X) \colon = (a_0 + b_0) 1 + (a_1 + b_1) X + \cdots + (a_n + b_n) X^n
            \end{align*}
        \item Puis :
            \begin{align*}
                (\sum_{k = 0}^m a_k X^k) \cdot (\sum_{l = 0}^{n} b_l X^l) \colon = \sum_{k = 0}^m \sum_{l = 0}^n a_k b_l X^{k + l}
            \end{align*}
\end{enumerate}

\begin{graybox}
    \begin{proposition}
        Soient $P, Q, R$ trois polynômes.   
        \begin{enumerate}
            \item $(P + Q) + R = P + (Q + R)$
            \item $P + 0 = 0 + P = P$
            \item $P + (-P) = (-P) + P = 0$
            \item $(P \cdot Q) \cdot R = P \cdot (Q \cdot R)$
            \item $P \cdot 1 = 1 \cdot P = P$
            \item $P \cdot Q = Q \cdot P$ 
            \item $(P + Q) R = P \cdot R + Q \cdot R$
            \item $P \cdot (Q + R) = P \cdot Q + P \cdot R$
        \end{enumerate}
    \end{proposition}
\end{graybox}

\subsection{Degré d'un polynôme}
\begin{graybox}
    \begin{definition}[Degré d'un polynôme]
        $P = a_0 + a_1 X + \cdots a_n X^n$ avec $(a_n \neq 0)$ $\in \K[X]$
        On définit $n$ comme étant le degré de $P$, on le note :
        \begin{align*}
            \mathrm{deg}(P) = n
        \end{align*}
        On peut aussi décrire le degré sous la forme d'une application :
        $\mathrm{deg} \colon \K[X] \to \N$
    \end{definition}
\end{graybox}

\begin{graybox}
    \begin{proposition}[Propriétés sur les degrés]~ 
        \begin{enumerate}
            \item Soit $\lambda \in \K$ on a : $\mathrm{deg}(\lambda) = 0$
            \item Soient $P$ et $Q$ deux polynômes on a : $\mathrm{deg}(P \cdot Q) = \mathrm{deg}(P) + \mathrm{deg}(Q)$
            \item Soient $P$ et $Q$ deux polynômes on a :
            $\mathrm{deg}(P + Q) = \mathrm{max}\left(\mathrm{deg}(P),\ \mathrm{deg}(Q)\right)$
        \end{enumerate}
    \end{proposition}
\end{graybox}

\begin{graybox}
    \begin{theoreme}[Division euclidienne]
        $P_1, P_2$ deux polynômes non nuls.
        \begin{align*}
            \exists ! (Q, R) \in \K[X]^2 \text{ tel que } P_1 = P_2 Q + R 
        \end{align*}
        Vérifiant :
             $\mathrm{deg}(R) = -\infty$ ou $0 \leq \mathrm{deg}(R) < \mathrm{deg}(Q)$
    \end{theoreme}
\end{graybox}

\begin{remarque}
Certaines propriétés vues dans le chapitre précédent, sont analogues avec les polynômes.
\end{remarque}

\begin{exemple}
$P_1 = X^3 + 6X^2 + 11X + 6$ et $P_2 = X + 3$
\begin{align*}
X^3 + 6X^2 + 11X + 6 = (X+3)(X^2 + 13X + 2)  
\end{align*}
\end{exemple}

\begin{graybox}
\begin{definition}[Polynôme irréductible]
Un polynôme $P \in \K[X]$ \textbf{non constant} est dit \textbf{irréductible}, s'il n'existe pas $(P_1, P_2) \in \K^[X]$ tel que $P = P_1 P_2$ et $\mathrm{deg}(P_i) < \mathrm{deg}(P)$
\end{definition}
\end{graybox}

\begin{graybox}
\begin{proposition}
Soit $P \in \K[X]$ \textbf{non constant et irréductible}. On a les propriétés suivantes :
\begin{enumerate}
\item $\K = \C \iff \mathrm{deg}(P) = 1$
\item $\K = \R \iff \text{ soit } \mathrm{deg}(P) = 1 \text{ ou bien } \mathrm{deg}(P) = 2$ avec le discriminant négatif.
\end{enumerate}
\end{proposition}
\end{graybox}

\begin{graybox}
\begin{proposition}
Soit $P(X)$ un polynôme \textbf{non constant}, si $\alpha$ est une racine de $P(X)$ alors on a :
\begin{align*}
X - \alpha \mid P(X)
\end{align*}
Autrement dit :
\begin{align*}
P(\alpha) = 0 \iff X - \alpha \mid P(X)
\end{align*}
\end{proposition}
\end{graybox}

\begin{proof}
Division euclidienne de $P$ par $X - \alpha$.
\begin{align*}
\exists ! (Q, R) \in \K[X] \times \K \text{ tel que } \\
P(X) = (X - \alpha) Q(X) + R
\end{align*}
On a :
\begin{align*}
P(\alpha) = (\alpha - \alpha) Q(\alpha) + R = R
\end{align*}
Puisque par hypothèse $P(\alpha) = 0$, le reste $R$ est nul.
Ainsi il reste
\begin{align*}
P(X) = (X - \alpha) Q(X)
\end{align*}
Ce qui revient à dire que $X - \alpha$ divise $P(X)$
\end{proof}

\begin{graybox}
\begin{corollaire}
Un polynôme de degré $d$ a au plus $d$ racines.
\end{corollaire}
\end{graybox}

\begin{graybox}
\begin{proposition}
Soit $P \in \K[X]$ non nul. Soit $\alpha \in \C$.
\begin{align*}
P(\alpha) = 0 \iff P(\overline{\alpha}) = 0
\end{align*}
\end{proposition}
\end{graybox}

\begin{proof}
$P(X) = a_0 + a_1X + \cdots + a_p X^p,\ a_i \in \R$
\begin{align*}
\overline{P(\alpha)} &= \overline{a_0 + a_1\alpha + \cdots + a_p \alpha^p} \\
&= \overline{a_0} + \overline{a_1\alpha} + \cdots + \overline{a_p \alpha^p} \\
&= \overline{a_0} + \overline{a_1}(\overline{\alpha}) + \cdots + \overline{a_p}(\overline{\alpha})^p \\
&= a_0 + a_1 (\overline{\alpha}) + \cdots + a_p (\overline{\alpha})^p \\
&= P(\overline{\alpha})
\end{align*}
d'où $P(\alpha) = 0 \iff P(\overline{\alpha}) = 0$
\\
On a donc $X - \alpha \mid P$ et $X - \overline{\alpha} \mid P$ \\
Ainsi 
\begin{align*}
X - \alpha \mid P,\ \exists Q \text{ tel que } P(X) = (X - \alpha)Q(X)
\end{align*}
\begin{align*}
\overline{\alpha} \in \C \backslash \R \implies \alpha \neq \overline{\alpha}, \ X - \overline{\alpha} \mid P \iff X - \overline{\alpha} \mid (X - \alpha) Q(X) \\
\mathrm{pgcd}(X - \alpha, X - \overline{\alpha}) = 1
\end{align*}
Par le Lemme de Gauss :
\begin{align*}
X - \overline{\alpha} \mid Q \\
\exists Q_1 \text{ tel que } Q_1(X) = (X - \overline{\alpha})Q_1(X)
\end{align*}
\begin{align*}
P(X) &= (X - \alpha) Q(X) \\
	 &= (X - \alpha)(X - \overline{\alpha}) Q_1(X) \\
	 &= X^2 - (\alpha + \overline{\alpha})X + \alpha \overline{\alpha} \\
	 &= X^2 - 2Re(\alpha)X + |\alpha|^2
\end{align*}
\end{proof}

\begin{graybox}
\begin{definition}
Soit $P$ un polynôme non constant. Soit $\alpha \in \K$ et $m \in \N^*$ $\alpha$ est une racine d'ordre $m$ de $P$ si et seulement si.
\begin{align*}
(X - \alpha)^m \mid P \text{ et } (X - \alpha)^{m+1} \nmid P
\end{align*}
\end{definition}
\end{graybox}

\begin{graybox}
\begin{theoreme}
Soit $P$ un polynôme non constant, $\alpha \in \K$, $m \in \N^*$
\begin{align*}
(X - \alpha)^m \mid P \iff P(\alpha) = P'(\alpha) = \cdots = P^{m-1}(\alpha) = 0
\end{align*}
\end{theoreme}
\end{graybox}

\begin{proof} Pour $m = 2$
$\implies$ 
\begin{align*}
Q \in \K [X] \text{ tel que } P(X) = (X - \alpha)^2 Q(X) \\
P(\alpha) = (\alpha - \alpha)^2 Q(\alpha) = 0 \\
P(X) = 2(X - \alpha) Q(X) + (X - \alpha)^2 Q(X) \\
P(\alpha) = 2 (\alpha - \alpha) Q(\alpha) + (\alpha - \alpha)^2 Q(\alpha) = 0
\end{align*}
$\impliedby$
\begin{align*}
P(\alpha) = 0 \implies \exists Q \in \K[X] \text{ tel que } P(X) = (X - \alpha) Q(X) \\
P'(X) = Q(X) + (X - \alpha)Q'(X) \\
P'(\alpha) = Q(\alpha) + (\alpha - \alpha) Q'(\alpha) = Q(\alpha) = 0
\end{align*}
\end{proof}

\begin{remarque}
Regarder dans le TD 6, les exercices 3, 5, 7, 12, 14, 15, 16, 19, 21
\end{remarque}

\begin{graybox}
\begin{theoreme}[Théorème fondamental de l'algèbre]
Soit $P(X)$ un polynôme à coefficients complexes de degré $n$. $P(X)$ admet $n$ racines complexes. 
\end{theoreme}
\end{graybox}

\begin{remarque}
Ce théorème s'appelle également le théorème d'Alembert-Gauss.
\end{remarque}
