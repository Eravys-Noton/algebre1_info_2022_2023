\chapter{Arithmétique}
\section{Divisibilité}
\begin{definition}
Soient $a \in \Z, b \in \Z^*$. 
On dit que :
\begin{itemize}
\item $a$ est un multiple de $b \iff \exists q \in \Z, a = bq$
\item $b$ est un diviseur de $a \iff \exists q \in \Z, a = bq$
\item $b$ divise $a$ $\iff b \mid a$
\end{itemize}
\end{definition}

\begin{theoreme}[Division euclidienne]
Soit $(a, b) \in \Z \times \Z^*$. Alors
\begin{align*}
\exists ! (q, r) \in \Z^2, a = bq + r,\ (0 \leq r < |b|)
\end{align*}
\textbf{Vocabulaire :}
\begin{itemize}
\item $a$ est appelé le \textit{dividende}
\item $b$ est appelé le \textit{diviseur}
\item $q$ est appelé le \textit{quotient}
\item $r$ est appelé le \textit{reste}
\end{itemize}
\end{theoreme}

\begin{proof}
Nous devons montrer deux choses, \textit{l'existence} et \textit{l'unicité} du couple $(q, r)$
\begin{enumerate}
\item \textbf{Existence}
\\
Supposons $a \in \N$ et considérons $M = \left\{ n \in \N \mid nb \leq a \right\}$ l'ensemble des multiples de $b$ inférieurs à $a$. $M$ est une partie de $\N$.
Nous avons deux propriétés :
\begin{enumerate}
\item $M$ est non vide car $0$ est un multiple de $b$ inférieur à $a$
\item $M$ est majoré par $a$ d'après sa définition.
\end{enumerate}
Ainsi, $M$ admet un plus grand élément que l'on notera $q$, vérifiant :
\begin{enumerate}
\item $qb \leq a$ car $q \in M$
\item $(q+1) b > a$ car $q + 1 > q$ sachant que $q$ est le plus grand élément de $M$, $q + 1 \notin M$.
\end{enumerate}
Posons $r = a - bq \Longleftrightarrow a = bq + r$.
Sachant que $a \geq bq$, $r \geq 0$. 
\\
On a $r < b$ car $b = (q+1)b - qb > a - bq = r$.
\\
Supposons que $a \in \Z$. Si $a$ est positif, on se ramène au cas précédent.
\\
Dans le cas où $a < 0$, $-a \geq 0$, ainsi, $\exists (q', r') \in \Z^2$ tel que 
\begin{align*}
-a &= bq' + r' \text{ avec } 0 \leq r' < |b| \\
\iff a &= b(-q') -r'
\end{align*}
Si $r' = 0$, on pose $q = -q'$ et $r = 0$ et on obtient le couple recherché. 
\\
Si $r' \neq 0$, $r' \in \llbracket 1, b - 1 \rrbracket$  et $a = b(-q' - 1) + (b - r')$, on pose $q = -q' - 1$ et $r = b - r'$ et on obtient le couple recherché.
\item \textbf{Unicité} \\
Pour cette partie, il suffit de supposer deux couples $(q, r) \in \Z^2$ et $(q', r') \in \Z^2$ et de montrer que $q = q'$ et $r = r'$.
\\
Commençons par $a = bq + r,\ (0 \leq r < |b|)$ et $a = bq' + r',\ (0 \leq r' < |b|)$. Comme $0 \leq r < b$ et $0 \leq r' < b$, on a : 
\begin{align*}
b|q' - q| = |r' - r| < b
\end{align*}
ce qui n'est possible que si $|q' - q| = 0$ ce qui implique que $q = q'$. Ceci entraîne donc $r = r'$ et donc on a montré que $(q, r) = (q', r')$
\end{enumerate}
\end{proof}

\section{PGCD et PPCM}
\begin{definition}[PPCM et PGCD]
Soit $(a, b) \in \Z^2 \backslash \{(0, 0)\}$ tel que $ab \neq 0$
\begin{align*}
\mathcal{M} \colon = \left\{ m \in \Z \ | \ a \mid m \text{ et } b \mid m \right\} \Rightarrow \mathcal{M} \neq \phi \text{ car } ab \in \mathcal{M}
\end{align*}
\begin{align*}
\mathcal{M} \cap \N^* \leftarrow \text{ il y a le plus petit commun multiple (PPCM) }
\end{align*}
\begin{align*}
\mathcal{D} = \left\{ d \in \Z \mid d \mid a \text{ et } d \mid b \right\} \Rightarrow \mathcal{D} \neq \emptyset \text{ car } 1 \in \mathcal{D}
\end{align*}
On a : $d \mid a, b \implies |d| \leq m \min (|a|, |b|)$ et $\mathrm{Card}(\mathcal{D}) < \infty$
Il y a le plus grand élément $\leftarrow$ le plus grand commun diviseur (PGCD)
\end{definition}

\begin{theoreme}[PPCM]
Soit $(a, b) \in \Z^* \times \Z^*$ et $m \in \Z,\ a \mid m \text{ et } b \mid m$. Alors $\mathrm{ppcm}(a, b) \mid m$
\end{theoreme}

\begin{proof}
Posons $\ell = \mathrm{ppcm}(a, b)$
\begin{align*}
\exists ! (q, r) \in \Z^2, \ m &= q\ell + r,\ 0 \leq r < \ell \\
\iff r &= m - q\ell,\ m \text{ et } \ell \text{ sont multiples de } a \text{ et } 
\\
&r \text{ est aussi un multiple de } a \text{ et } b
\end{align*}
Par la minimalité de $\ell$, $r = 0 \implies m = q \ell$
\end{proof}

\begin{theoreme}[PGCD]
Soit $(a, b) \in \Z^* \times \Z^*$ et $d \in \Z,\ d \mid a \text{ et } d \mid b$. Alors $d \mid \mathrm{pgcd}(a, b)$
\end{theoreme}

\begin{proof}
Posons $m = \mathrm{pgcd}(a, b)$. Il suffit de montrer que 
\begin{align*}
\mathrm{pgcd}(m, d) = m
\end{align*}
Soit $\ell = \mathrm{ppcm}(m ,d)$, $\ell \geq m$, $a$ et $b$ sont multiples de $m$ et $d$
\\
D'après le théorème précédent : 
\begin{align*}
\ell \mid a \text{ et } \ell \mid b, \ l \leq m
\end{align*}
Sachant qu'on a $\ell \geq m$ et $\ell \leq m$, on en conclut que $\ell = m$
\end{proof}

\begin{theoreme}
Soit $(a, b) \in (\N^*)^2 \implies ab = \mathrm{pgcd}(a, b)\mathrm{ppcm}(a, b)$
\end{theoreme}

\begin{definition}[Nombres premiers entre eux]
	Soit $(a, b) \in (\Z^*)^2$
	\begin{align*}
	a \text{ et } b \text{ premiers entre eux } \iff \mathrm{pgcd}(a, b) = 1
	\end{align*}
\end{definition}

\section{Algorithme d'Euclide}
\begin{proposition}[Algorithme d'Euclide]
Soient $a \in \Z^*$, $b \in \Z^*$ tel que 
\begin{align*}
|a| > |b| \implies \exists ! (q, r) \in \Z^2, a = b q + r, 0 \leq r < |b| \\
\mathrm{pgcd}(a, b) = \mathrm{pgcd}(b, a) = \mathrm{pgcd}(b, a - qb) = \mathrm{pgcd}(b, r) \\
\end{align*}
\begin{equation*}
\mathrm{pgcd}(a, b) = \mathrm{pgcd}(b, r)
\end{equation*}
Si $r = 0 \implies a = q b,\ \mathrm{pgcd}(a, b) = b$ 
Supposons que $r = 0$ :
\begin{equation*}
\exists ! (q_1, r_1),\ b = q_1 r + r_1, \ 0 \leq r_1 < r
\end{equation*}
Si $r_1 \neq 0 \implies \exists ! (q_2, r_2),\ r = q_2 r_1 + r_2,\ 0 \leq r_2 < r_1$ \\
$\vdots$ \\
Si $r_{n - 2} \neq 0 \implies \exists ! (q_{n - 1}, r_{n - 1}),\ r_{n - 3} = q_{n-1}r_{n-2} + r_{n - 1},\ 0 \leq r_{n-1} < r_{n-2}$

\begin{equation*}
\exists q_n,\ r_{n-2} = q_n r_{n-1}
\end{equation*}
\begin{align*}
\mathrm{pgcd}(a, b) &= \mathrm{pgcd}(b, r) \\
				   &= \mathrm{pgcd}(r, r_1) \\
				   &= \mathrm{pgcd}(r_1, r_2) \\
				   &\vdots \\
				   &= \mathrm{pgcd}(r_{n-2}, r_{n-1}) \\
				   &= \mathrm{pgcd}(q_n r_{n-1}, r_{n-1}) = r_{n-1}
\end{align*}
\end{proposition}

\begin{exemple}~ 
\begin{enumerate}
\item $\mathrm{pgcd}(72, 58)$
\begin{align*}
72 &= 58 \times 1 + 14 \\
58 &= 14 \times 4 + 2 \\
14 &= 2 \times 7 + 0
\end{align*}
On en conclut que $\mathrm{pgcd}(72, 58) = 2$

\item $\mathrm{pgcd}(625, 216)$
\begin{align*}
625 &= 216 \times 2 + 193 \\
216 &= 193 \times 1 + 23 \\
193 &= 23 \times 8 + 9 \\
23 &= 9 \times 2 + 5 \\
9 &= 5 \times 1 + 4 \\
5 &= 4 \times 1 + 1
\end{align*}
On en conclut que $\mathrm{pgcd}(625, 216) = 1$
\end{enumerate}
\end{exemple}