\chapter{Arithmétique}
\section{Divisibilité}
\begin{definition}
Soient $a \in \Z, b \in \Z^*$. 
On dit que :
\begin{itemize}
\item $a$ est un multiple de $b \iff \exists q \in \Z, a = bq$
\item $b$ est un diviseur de $a \iff \exists q \in \Z, a = bq$
\item $b$ divise $a$ $\iff b \mid a$
\end{itemize}
\end{definition}

\begin{theoreme}[Division euclidienne]
Soit $(a, b) \in \Z \times \Z^*$. Alors
\begin{align*}
\exists ! (q, r) \in \Z^2, a = bq + r,\ (0 \leq r < |b|)
\end{align*}
\end{theoreme}

\begin{proof}
Soit $b > 0$, $\Z = \displaystyle{\bigcup_{k \in \Z}^{}} ([kb, kb + |b|[ \cap \Z)$ On peut écrire la réunion disjointe $\displaystyle{\bigsqcup_{k \in \Z}^{}}$
Alors 
\begin{align*}
\exists ! q \in \Z,\ a \in [qb, qb + |b|[ \iff qb \leq a < qb + |b|
\\
0 \leq r = a - qb < |b|
\end{align*}
\begin{align*}
\bigsqcup_{\lambda \in A} A_{\lambda} = \bigcup_{k \in A} A_{\lambda}
\end{align*}

\begin{align*}
A_{\lambda} \cap A_{\mu} = \phi,\ \lambda \neq \\
[2k, 2k+2[,\ b = 2, \ k = 1, 2 \\
[2, 4[ \cap \Z = \{2,  3\}
\\
[4, b[ \cap \Z = \{4, 5\}
\end{align*}
\end{proof}

\section{PGCD et PPCM}
\begin{definition}[PPCM et PGCD]
Soit $(a, b) \in \Z^2 \backslash \{(0, 0)\}$ tel que $ab \neq 0$
\begin{align*}
\mathcal{M} \colon = \left\{ m \in \Z \ | \ a \mid m \text{ et } b \mid m \right\} \Rightarrow \mathcal{M} \neq \phi \text{ car } ab \in \mathcal{M}
\end{align*}
\begin{align*}
\mathcal{M} \cap \N^* \leftarrow \text{ il y a le plus petit commun multiple (PPCM) }
\end{align*}
\begin{align*}
\mathcal{D} = \left\{ d \in \Z \mid d \mid a \text{ et } d \mid b \right\} \Rightarrow \mathcal{D} \neq \emptyset \text{ car } 1 \in \mathcal{D}
\end{align*}
On a : $d \mid a, b \implies |d| \leq m \min (|a|, |b|)$ et $\mathrm{Card}(\mathcal{D}) < \infty$
Il y a le plus grand élément $\leftarrow$ le plus grand commun diviseur (PGCD)
\end{definition}

\begin{theoreme}[PPCM]
Soit $(a, b) \in \Z^* \times \Z^*$ et $m \in \Z,\ a \mid m \text{ et } b \mid m$. Alors $\mathrm{ppcm}(a, b) \mid m$
\end{theoreme}

\begin{proof}
Posons $\ell = \mathrm{ppcm}(a, b)$
\begin{align*}
\exists ! (q, r) \in \Z^2, \ m &= q\ell + r,\ 0 \leq r < \ell \\
\iff r &= m - q\ell,\ m \text{ et } \ell \text{ sont multiples de } a \text{ et } 
\\
&r \text{ est aussi un multiple de } a \text{ et } b
\end{align*}
Par la minimalité de $\ell$, $r = 0 \implies m = q \ell$
\end{proof}

\begin{theoreme}[PGCD]
Soit $(a, b) \in \Z^* \times \Z^*$ et $d \in \Z,\ d \mid a \text{ et } d \mid b$. Alors $d \mid \mathrm{pgcd}(a, b)$
\end{theoreme}

\begin{proof}
Posons $m = \mathrm{pgcd}(a, b)$. Il suffit de montrer que 
\begin{align*}
\mathrm{pgcd}(m, d) = m
\end{align*}
Soit $\ell = \mathrm{ppcm}(m ,d)$, $\ell \geq m$, $a$ et $b$ sont multiples de $m$ et $d$
\\
D'après le théorème précédent : 
\begin{align*}
\ell \mid a \text{ et } \ell \mid b, \ l \leq m
\end{align*}
Sachant qu'on a $\ell \geq m$ et $\ell \leq m$, on en conclut que $\ell = m$
\end{proof}

