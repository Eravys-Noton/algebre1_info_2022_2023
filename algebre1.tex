\documentclass[a4paper, 11pt]{report}

\usepackage[utf8]{inputenc}
\usepackage[margin=2.5cm]{geometry}
\usepackage{amsmath, amsfonts, amsthm, amssymb, mathtools}
\usepackage[french]{babel}
\usepackage[T1]{fontenc}
\usepackage{framed}
\usepackage[listings, skins, breakable, hooks]{tcolorbox}
\usepackage{indentfirst}
\usepackage{hyperref, xcolor, color}
\usepackage{graphicx}
\usepackage{diagbox}
\usepackage{stmaryrd}
\usepackage{caption}
\usepackage{subcaption}
\usepackage{indentfirst}
\usepackage[Lenny]{fncychap}


\usepackage{tikz}
\usepackage{pgfplots}
\pgfplotsset{compat=1.11}
\usepackage{mathrsfs}
\usetikzlibrary{arrows}
\pagestyle{empty}

\usepackage{fancyhdr}
\usepackage{lastpage}

\fancypagestyle{fancy}{
\fancyhf{}
\rhead{Algèbre 1 - Informatique}
\lhead{Année scolaire 2022-2023}
\renewcommand{\footrulewidth}{0.25pt}
\cfoot{\thepage \  / \pageref{LastPage}}
}

\fancypagestyle{plain}{
\fancyhf{}
\cfoot{\thepage \  / \pageref{LastPage}}
\renewcommand{\headrulewidth}{0pt}
\renewcommand{\footrulewidth}{0.25pt}
}

\theoremstyle{definition}
\newtheorem{definition}{Définition}[section]
\newtheorem{proposition}{Proposition}[section]
\newtheorem{theoreme}{Théorème}[section]
\newtheorem{lemme}{Lemme}[section]
\newtheorem{corollaire}{Corollaire}[section]
\newtheorem{remarque}{Remarque}[section]
\newtheorem{exemple}{Exemple}[section]

\newcommand{\N}{\mathbb{N}}
\newcommand{\Z}{\mathbb{Z}}
\newcommand{\Q}{\mathbb{Q}}
\newcommand{\R}{\mathbb{R}}
\newcommand{\C}{\mathbb{C}}
\newcommand{\K}{\mathbb{K}}

\newtcolorbox{graybox}{breakable, enhanced, colframe=white, colback=gray!20!white, arc=0mm, outer arc=0mm}

\DeclareMathSymbol{\mlqq}{\mathrel}{operators}{"5C}
\DeclareMathSymbol{\mrqq}{\mathrel}{operators}{`"}

\renewcommand{\leq}{\leqslant}
\renewcommand{\geq}{\geqslant}

\title{Algèbre 1 pour les informaticiens}
\date{Année scolaire 2022-2023}
\author{}

\begin{document}

	\definecolor{xdxdff}{rgb}{0.49019607843137253,0.49019607843137253,1}
\definecolor{qqwuqq}{rgb}{0,0.39215686274509803,0}
\definecolor{qqqqff}{rgb}{0,0,1}
	
	\maketitle
	\tableofcontents
	\clearpage
	\pagestyle{fancy}
	\chapter{Calcul Algébrique}
\section{Point sur les ensembles de nombres}

\begin{definition}[Ensemble des nombres entiers naturels]
	\begin{align*}
		\N = \{0; 1; ...\}
	\end{align*}
\end{definition}

\begin{definition}[Ensemble des nombres entiers relatifs]
	\begin{align*}
		\Z = \{...; -1; 0; 1; ...\}
	\end{align*}
\end{definition}

\begin{definition}[Ensemble des nombres rationnels]
	\begin{align*}
		\Q = \left\{\frac{a}{b} \mid a \in \Z, b \in \N^* \right\}
	\end{align*}
\end{definition}

\begin{definition}[Ensemble des nombres réels]
	\begin{align*}
		\R = ]-\infty; +\infty[
	\end{align*}
\end{definition}

\subsection{Axiomatique}
Ici $\K$ désigne soit $\N$, soit $\Z$, soit $\Q$, soit $\R$
\begin{proposition}[Loi de composition +]~
	\begin{enumerate}
		\item Associativité :
		\begin{align*}
			\forall(a, b, c) \in \K^3,\ a + (b + c) = (a + b) + c
		\end{align*}
		\item Commutativité :
		\begin{align*}
			\forall(a, b) \in \K^2,\ a + b = b + a
		\end{align*}
		\item Existence d'un élément neutre : 
		\begin{align*}
			\forall a \in \K,\ a + 0 = a
		\end{align*}
		\item Symétrie :
		\begin{align*}
			\forall(a, a') \in \K^2,\ a + a' = 0 \textnormal{ avec } a' = -a
		\end{align*}
		Remarque : Cette propriété ne s'applique pas dans $\N$
	\end{enumerate}
\end{proposition}
\clearpage
\begin{proposition}[Loi de composition $\cdot$]~
	\begin{enumerate}
		\item Associativité :
		\begin{align*}
			\forall (a, b, c) \in \K^3,\ (a \cdot b) \cdot c = a \cdot (b \cdot c)
		\end{align*}
		\item Commutativité :
		\begin{align*}
			\forall (a, b) \in \K^3,\ a \cdot b = b \cdot a
		\end{align*}
		\item Existence d'un élément neutre :
		\begin{align*}
			\forall a \in \K, \ & a \cdot 1 = a \\
			& a \cdot 0 = 0
		\end{align*}
		\item Distributivité : 
		\begin{align*}
			\forall (a, b, c) \in \K^3,\ &a \cdot (b + c) = a \cdot b + a \cdot c \\
			&(a + b) \cdot c = a \cdot c + b \cdot c
		\end{align*}
	\end{enumerate}
\end{proposition}
\section{Opérations sur les fractions}
\begin{proposition}[Addition sur les fractions] 
	\begin{align*}
		\forall (a, b, c, d) \in \Z^4,\ \frac{a}{b} + \frac{c}{d} = \frac{ad + bc}{bd}
	\end{align*}
\end{proposition}
\begin{proof}~
	\\
	$\forall (a, b, c, d, a', b', c', d') \in \Z^8$
	
	\noindent D'après la proposition on a :
	\begin{align*}
		\frac{a}{b} + \frac{c}{d} = \frac{ad + bc}{bd}
	\end{align*}
	et :
	\begin{align*}
		\frac{a'}{b'} + \frac{c'}{d'} = \frac{a'd' + b'c'}{b'd'}
	\end{align*}
	Montrons que : 
	\begin{align*}
		\frac{ad + bc}{bd} = \frac{a'd' + b'c'}{b'd'}
	\end{align*}
	On suppose que : 
	\begin{align*}
		\frac{a}{b} = \frac{a'}{b'} \iff a'b = ab' \\
		\frac{c}{d} = \frac{c'}{d'} \iff c'd = cd'
	\end{align*}
	On aurait donc :
	\begin{align*}
		&\frac{ad + bc}{bd} = \frac{a'd' + b'c'}{b'd'} \\
		\iff &(ad + bc)b'd' = bd(a'd' + b'c') \\
		\iff &(ad + bc)b'd' - bd(a'd' + b'c') = 0
	\end{align*}
	\begin{align*}
		(ad + bc)b'd' - bd(a'd' + b'c') &= (adb'd' + bcb'd') - (bda'd' + bdb'c')\\
		&= adb'd' + bcb'd' - bda'd' - bdb'c' \\
		&= adb'd' - a'd'bd + bcb'd' - b'c'bd \\
		&= ab'dd' - a'bdd' + cd'bb' - c'dbb' \\
		&= (ab' - a'b)dd' + (cd' - c'd)dd'
	\end{align*}
	D'après l'hypothèse de départ :
	\begin{align*}
		ab' = a'b \iff ab' - a'b = 0 \\
		cd' = c'd \iff cd' - c'd = 0
	\end{align*}
	Donc : 
	\begin{align*}
		(\underbrace{ab' - a'b}_{0})dd' + (\underbrace{cd' - c'd}_0)dd' = 0
	\end{align*}
	On obtient alors :
	\begin{align*}
		(ad + bc)b'd' - bd(a'd' + b'c') = 0
	\end{align*}
\end{proof}
\begin{proposition}[Multiplication sur les fractions]
	\begin{align*}
		\forall (a, b, c, d) \in \Z^4,\ \frac{a}{b} \times \frac{c}{d} = \frac{ac}{bd}
	\end{align*}
\end{proposition}
\begin{proof}~
	\\
	$\forall (a, b, c, d, a', b', c', d') \in \Z^8$\\
	D'après la proposition on a :
	\begin{align*}
		\frac{a}{b} \cdot \frac{c}{d} = \frac{ac}{bd}
	\end{align*}
	et :
	\begin{align*}
		\frac{a'}{b'} \cdot \frac{c'}{d'} = \frac{a'c'}{b'd'}
	\end{align*}
	Montrons que :
	\begin{align*}
		\frac{ac}{bd} = \frac{a'c'}{b'd'}
	\end{align*}
	On suppose que : 
	\begin{align*}
		\frac{a}{b} = \frac{a'}{b'} \iff a'b = ab' \\
		\frac{c}{d} = \frac{c'}{d'} \iff c'd = cd'
	\end{align*}
	On aurait donc :
	\begin{align*}
		&\frac{ac}{bd} = \frac{a'c'}{b'd'} \\
		\iff &acb'd' = bda'c' \\
		\iff &acb'd' - bda'c' = 0
	\end{align*}
	\begin{align*}
		acb'd' - bda'c' &= (ab')(cd') - (a'b)(c'd) \\
		&= (ab')(cd') - (a'b)(cd') + (a'b)(cd') - (a'b)(c'd) \\
		&= (ab' - a'b)(cd') + (cd' - c'd)(a'b)
	\end{align*}
	D'après l'hypothèse de départ :
	\begin{align*}
		ab' = a'b \iff ab' - a'b = 0 \\
		cd' = c'd \iff cd' - c'd = 0
	\end{align*}
	Donc :
	\begin{align*}
		(\underbrace{ab' - a'b}_0)(cd') + (\underbrace{cd' - c'd}_0)(a'b) = 0
	\end{align*}
	On obtient alors :
	\begin{align*}
		acb'd' - bda'c' = 0
	\end{align*}
\end{proof}
\section{Sommes}
\begin{definition}[Définition de la somme]
	$\forall m, n \in \N \textnormal{ avec } m \leq n \textnormal{ et } a_k \in \R, \ m \leq k \leq n$
	\begin{align*}
		\sum_{k = m}^{n}a_k = a_m + a_{m+1} + \cdots + a_n
	\end{align*}
\end{definition}

\begin{remarque}
	L'indice de sommation est important car :
	\begin{align*}
		\sum_{k = m}^{n}a_l = \underbrace{a_l + a_l + \cdots + a_l}_{n - m + 1 \textnormal{termes}} = (n - m + 1)a_l
	\end{align*}
\end{remarque}
\begin{proposition}[Linéarité de la somme]
	$\forall m, n \in \N \textnormal{ avec } m \leq n$ et $a_k, b_k \in \R, \ m \leq k \leq n$
	\begin{align*}
		\sum_{k = m}^{n} (a_k + b_k) = \sum_{k=m}^{n}a_k + \sum_{k=m}^{n}b_k
	\end{align*}
\end{proposition}
\begin{proof}
	$\forall m, n \in \N \textnormal{ avec } m \leq n \textnormal{ et } a_k, b_k \in \R, \ m \leq k \leq n$
	\begin{align*}
		\sum_{k = m}^{n} (a_k + b_k) &= (a_m + b_m) + (a_{m+1} + b_{m+1}) + \cdots (a_n + b_n)\\
		&= (a_m + a_{m+1} + \cdots + a_n) + (b_m + b_{m+1} + \cdots + b_n) \\
		&= \sum_{k=m}^{n}a_k + \sum_{k=m}^{n}b_k
	\end{align*}
\end{proof}
\begin{proposition}[Linéarité de la multiplication de la somme par une constante]~
	\\
	$\forall m, n \in \N \textnormal{ avec } m \leq n \textnormal{ et } a_k, \lambda \in \R, \ m \leq k \leq n$
	\begin{align*}
		\sum_{k = m}^{n} (\lambda a_k) = \lambda \sum_{k = m}^{n} a_k
	\end{align*}
\end{proposition}
\begin{proof}
	$\forall m, n \in \N \textnormal{ avec } m \leq n \textnormal{ et } a_k, \lambda \in \R, \ m \leq k \leq n$
	\begin{align*}
		\sum_{k = m}^{n} (\lambda a_k) &=  \lambda a_m + \lambda a_{m+1} + \cdots + \lambda a_n \\
		&= \lambda (a_m + a_{m+1} + \cdots + a_n)
	\end{align*}
\end{proof}
\subsection{Quelques sommes importantes}
\begin{enumerate}
	\item \label{1.3.1_sum1} $\displaystyle{\sum_{k = 1}^{n} k = \frac{n(n+1)}{2}}$ avec $n \in \N$
	\item \label{1.3.1_sum2} $\displaystyle{\sum_{k = 1}^{n}\left(a + \left(k - 1\right)d\right) = \frac{1}{2}n\left(2a + \left(n - 1\right)d\right)}$ avec $n \in \N \text{ et } a, d \in \R$
	\item \label{1.3.1_sum3} $\displaystyle{\sum_{k = 0}^{n-1}ar^k = \sum_{k = 1}^{n}ar^{k-1} = a \cdot \frac{1 - r^n}{1 - r}}$ avec $n \in \N \text{ et } a, r \in \R$
\end{enumerate}
\begin{proof}
	\ref{1.3.1_sum1} On pose S = $\displaystyle{\sum_{k = 1}^{n} k}$ avec $n \in \N$ \\
	On a donc :
	\begin{align*}
		S = & 1 + 2 + \cdots + (n - 1) + n \\
		& n + (n - 1) + \cdots + 2 + 1 = S 
	\end{align*}
	En additionnant les termes du "dessus" et du "dessous" on obtient :
	\begin{align*}
		&2S = n \cdot (n+1) \\
		&S = \frac{n(n+1)}{2}
	\end{align*}
\end{proof}

\begin{proof}
	\ref{1.3.1_sum2} On pose S = $\displaystyle{\sum_{k = 1}^{n}\left(a + \left(k - 1\right)d\right)}$ avec $n \in \N,\ a, d \in \R$
	\begin{align*}
		S = \sum_{k = 1}^{n} \left(a + \left(k - 1\right)d\right) &= \sum_{k=1}^{n} (a - d + dk)\\
		&= \sum_{k=1}^{n} (a - d) + \sum_{k = 1}^{n} dk \\
		&= \sum_{k=1}^{n} (a - d) + d\sum_{k=1}^{n}k \\
		&= n(a - d) + d \frac{n(n + 1)}{2} \\
		&= n\left(\left(a - d\right) + \frac{d(n+1)}{2} \right) \\
		&= \frac{1}{2}n \left(2\left(a-d)\right) + d(n+1) \right) \\
		&= \frac{1}{2}n (2a - 2d + nd + d) \\
		&= \frac{1}{2}n (2a -d + nd) \\
		&= \frac{1}{2}n (2a + (n - 1)d)
	\end{align*}
\end{proof}

\begin{proof}
	\ref{1.3.1_sum3} On pose S = $\displaystyle{\sum_{k = 0}^{n-1}ar^k}$ avec $n \in \N,\ a, r \in \R$
	\begin{align*}
		&\begin{aligned}
			S &= a + ar + \cdots + ar^{n-1} \\
			rS &= ar + ar^2 + \cdots + ar^n 
		\end{aligned} 
		\\
		\\
		&\begin{aligned}
			S - rS = (a + &ar + \cdots + ar^{n-1}) \\
			- (&ar + \cdots + ar^{n-1} + ar^n)
		\end{aligned}
		\\
		\\
		&\begin{aligned}
			(1 - r)S &= a - ar^n \\
			S &= a \cdot \frac{1 - r^n}{1 - r}
		\end{aligned}
	\end{align*}
\end{proof}

\subsection{Sommes téléscopiques}
\begin{proposition}[Somme téléscopique]~
	$\forall m, n \in \N \textnormal{ avec }m \leq n,\ a_{k}  \in \R \textnormal{ avec } m \leq k \leq n$
	\begin{align*}
		\sum_{k = m}^{n} (a_k - a_{k-1}) = a_n - a_{m - 1} 
	\end{align*}
\end{proposition}

\begin{proof}
	$\forall m, n \in \N \textnormal{ avec }m \leq n,\ a_{k}  \in \R \textnormal{ avec } m \leq k \leq n$
	\begin{align*}
		&\begin{aligned}
			\sum_{k = m}^{n}(a_k + a_{k - 1}) = &(\underline{a_m} - a_{m-1}) \\
			+&(\underline{a_{m+1}} - \underline{a_m}) \\
			+&(a_{m+2} - \underline{a_{m+1}}) \\
			& \qquad \vdots \\
			+& (\underline{a_{n-1}} - \underline{a_{n-2}}) \\	
			+& (a_n - \underline{a_{n - 1}}) \\
		\end{aligned}
		\\
		&\begin{aligned}
			\sum_{k = m}^{n}(a_k + a_{k - 1}) = a_n - a_{m - 1}
		\end{aligned}
	\end{align*}
\end{proof}

\section{Puissances}
\begin{definition}[Puissance d'un réel]
	$\forall a \in \R,\ \forall n \in \N$
	\begin{align*}
		a^n = \underbrace{a \times a \times \cdots \times a}_{n fois}
	\end{align*}
\end{definition}
\clearpage
\begin{proposition}[Propriétés des puissances]
	$\forall a \in \R,\ \forall m, n \in \N$
	\begin{enumerate}
		\item \label{1.4_power1} $a^m \times a^n = a^{m + n}$
		\item \label{1.4_power2} $(a^m)^n = a^{mn}$
		\item \label{1.4_power3} $\displaystyle{\frac{a^n}{a^m}} = a^{n - m}, a \neq 0$
		\item \label{1.4_power4} $a^{-m} = \displaystyle{\frac{1}{a^m}}, a \neq 0$
		\item \label{1.4_power5} $a^0 = 1$
	\end{enumerate}
\end{proposition}

\begin{proof}
	\ref{1.4_power1} $\forall a \in \R,\ \forall n, m \in \N$
	\begin{align*}
		a^m \times a^n &= \underbrace{(a \times a \times \cdots \times a)}_{m fois} \times \underbrace{(a \times a \times \cdots \times a)}_{n fois} \\
		&= \underbrace{a \times a \times \cdots \times a}_{m + n fois}
	\end{align*}
\end{proof}

\begin{proof}
	\ref{1.4_power2} $\forall a \in \R,\ \forall n, m \in \N$
	\begin{align*}
		(a^m)^n = \underbrace{\underbrace{(a \times a \times \cdots \times a)}_{m fois} \times \underbrace{(a \times a \times \cdots \times a)}_{m fois} \times \cdots \times \underbrace{(a \times a \times \cdots \times a)}_{m fois}}_{n fois} = \underbrace{a \times a \times \cdots \times a}_{m \times n fois}
	\end{align*}
\end{proof}

\begin{proof}
	\ref{1.4_power3} $\forall a \in \R^*,\ \forall n, m \in \N$
	\begin{align*}
		\frac{a^n}{a^m} = \frac{\overbrace{a \times a \cdots \times a}^{n fois}}{\underbrace{a \times a \cdots \times a}_{m fois}} = \underbrace{a \times a \times \cdots \times a}_{n - m fois}
	\end{align*}
\end{proof}

\begin{proof}
	\ref{1.4_power4} $\forall a \in \R^*,\ \forall m \in \N$
	\begin{align*}
		a^{-m} &= a^{0 - m}  \\
		&= \frac{a^0}{a^m} \\
		&= \frac{1}{a^m}
	\end{align*}
\end{proof}

\begin{proof}
	\ref{1.4_power5} $\forall a \in \R$
	\begin{align*}
		&a^1 = a \\
		&a^0 = \frac{a}{a} = 1
	\end{align*}
\end{proof}

	\chapter{Ensembles et applications}
\section{Ensembles}

\begin{definitionbox}
    \begin{definition}[Définition intuitive d'un ensemble]
	Un ensemble E est une collection d'objets appelés éléments.
	Si E contient un élément $x$, on dit que $x$ appartient à E, noté $x \in E$ 
\end{definition}
\end{definitionbox}

\begin{definitionbox}
    \begin{definition}[Ensemble vide]
	L'ensemble vide noté $\emptyset$ est l'ensemble ne contenant aucun élément.
\end{definition}
\end{definitionbox}

\begin{definitionbox}
    \begin{definition}[Inclusion]
	\begin{align*}
		&\text{Un ensemble F est inclus dans un ensemble E } \iff \forall x \in F,\ x \in E. \\
		&\text{On note : } F \subset E \text{ On dit aussi que F est un sous-ensemble, une partie de E}
	\end{align*}
\end{definition}
\end{definitionbox}

\begin{definitionbox}
    \begin{definition}[Egalité d'ensembles]
	\begin{align*}
		\text{Deux ensembles E et F sont égaux} \iff E \subset F \text{ et } F \subset E
	\end{align*} 
\end{definition}
\end{definitionbox}

\begin{definitionbox}
\begin{definition}[Singleton]
	Un singleton est un ensemble de ne contenant qu'un seul élément (noté entre accolades).
\end{definition}
\end{definitionbox}

\begin{definitionbox}
    \begin{definition}[Réunion d'ensembles]
	Soient E et F deux ensembles.
	\begin{align*}
		E \cup F \text{ (lu E union F) } = \{\forall x,\ x \in E \text{ ou } x \in F \}
	\end{align*}
\end{definition}
\end{definitionbox}

\begin{definitionbox}
    \begin{definition}[Intersection d'ensembles]
	Soient E et F deux ensembles.
	\begin{align*}
		E \cap F \text{ (lu E inter F) } = \{\forall x,\ x \in E \text{ et } x \in F \}
	\end{align*}
\end{definition}
\end{definitionbox}

\begin{propositionbox}
    \begin{proposition}[Propriétés sur les ensembles] Soient A, B, C, E des ensembles
	\begin{enumerate}
		\item Associativité :
		\begin{align*}
			A \cup (B \cup C) = (A \cup B) \cup C \\
			A \cap (B \cap C) = (A \cap B) \cap C
		\end{align*}
		\item Elément neutre : 
		\begin{align*}
			A \cup \emptyset = A \\
            A \cap A = A
		\end{align*}
		\item Intersection d'un ensemble et d'une partie :
		\begin{align*}
			A \subset E \iff \ A \cap E = E \cap A = A
		\end{align*}
		\item Commutativité :
		\begin{align*}
			A \cup B = B \cup A \\
			A \cap B = B \cap A
		\end{align*}
		\item Distributivité :
		\begin{align*}
			A \cup (B \cap C) = (A \cup B) \cap (A \cup C) \\
			A \cap (B \cup C) = (A \cap B) \cup (A \cap C)
		\end{align*}
	\end{enumerate}
\end{proposition}
\end{propositionbox}

\begin{definitionbox}
    \begin{definition}[Complémentaire d'un ensemble]
	\begin{align*}
		E \backslash F = \{\forall x,\ x \in E \text{ et } x \notin F \}
	\end{align*}
\end{definition}
\end{definitionbox}

\begin{leftstroke}
\begin{remarque}Soient E, F des ensembles.
	\begin{itemize}
		\item $(E \backslash F) \subset E$
		\item $(E \backslash F) \cap F = \emptyset$
		\item $E \backslash F \neq F \backslash E$
	\end{itemize}
\end{remarque}
\end{leftstroke}

\begin{leftstroke}
\begin{remarque}Soient E et A des ensembles.
	\begin{align*}
		&A \subset E \\
		&A^C = E \backslash A \\
		&(A^C)^C = A
	\end{align*}
\end{remarque}
\end{leftstroke}

\begin{propositionbox}
    \begin{proposition}[Lois de Morgan] Soient A et B des ensembles.
	\begin{enumerate}
		\item \label{prop10_1} $(A \cup B)^C = A^C \cap B^C$
		\item \label{prop10_2} $(A \cap B)^C = A^C \cup B^C$
	\end{enumerate}
\end{proposition}
\end{propositionbox}
\begin{proof}
	\ref{prop10_1} \\
	Soient A et B des ensembles et $x$ un élément quelconque.
	\\
	\framebox{$\subset$}
	Par définition du complémentaire : 
	\begin{align*}
		x \in (A \cup B)^C \iff x \notin (A \cup B)
	\end{align*}
	$x \notin A$ car $A \subset (A \cup B)$ ce qui impliquerait que $x \in (A \cup B)$ et donc il y aurait une contradiction. On obtient une contradiction similaire si on suppose que $x \in B$. Ainsi on a $x \in A^C \text{ et } x \in B^C$, donc par la définition de l'intersection on a :
	\begin{align*}
		x \in (A^C \cap B^C)
	\end{align*}
	d'où :
	\begin{align*}
		(A \cup B)^C \subset (A^C \cap B^C)
	\end{align*}
	\framebox{$\supset$} Par définition de l'intersection :
	\begin{align*}
		x \in (A^C \cap B^C) &\iff x \in A^C \text{ et } x \in B^C \\
		&\iff x \notin A \text{ et } x \notin B \\
		&\iff x \in (A \cup B)^C
	\end{align*}
	d'où :
	\begin{align*}
		(A^C \cap B^C) \subset (A \cup B)^C
	\end{align*}
	Ainsi : 
	\begin{align*}
		(A \cup B)^C = A^C \cap B^C
	\end{align*}
\end{proof}

\begin{proof}
	\ref{prop10_2} 
	\\
	Soient A et B des ensembles et $x$ un élément quelconque. \\
	\framebox{$\subset$} Par définition du complémentaire :
	\begin{align*}
		x \in (A \cap B)^C &\iff x \notin (A \cap B) \\ 
		&\iff x \notin A \text{ et } x \notin B \\
		&\iff x \in A^C \text{ et } x \in B^C \\
		&\iff x \in (A^C \cap B^C)
	\end{align*}
	Sachant que :
	\begin{align*}
		(A^C \cap B^C) \subset (A^C \cup B^C)
	\end{align*}
	On a :
	\begin{align*}
		x \in (A^C \cap B^C) \implies x \in (A^C \cup B^C)
	\end{align*}
	d'où :
	\begin{align*}
		(A \cap B)^C \subset (A^C \cup B^C)
	\end{align*}
	\framebox{$\supset$} Par définition de la réunion :
	\begin{align*}
		x \in (A^C \cup B^C) &\iff x \in A^C \text{ ou } x \in B^C \\
		&\iff x \notin A \text{ ou } x \notin B \\
		&\iff x \notin (A \cap B) \\
		&\iff x \in (A \cap B)^C
	\end{align*}
	Ainsi : 
	\begin{align*}
		(A^C \cap B^C) \subset (A \cup B)^C
	\end{align*}
	Donc :
	\begin{align*}
		(A \cap B)^C = A^C \cup B^C
	\end{align*}
\end{proof}

\begin{definitionbox}
    \begin{definition}[Produit cartésien]
	Soient E et F des ensembles
	\begin{align*}
		&- E \times F = \{(x, y),\ x \in E,\ y \in F\} \\
		&- E \times E = E^2 \\
		&- E \times E \times E = E^3
	\end{align*}
\end{definition}
\end{definitionbox}
\section{Applications}
\begin{definitionbox}
    \begin{definition}[Application]
	Soient E et F deux ensembles. $f:E \to F$ est une application si pour chaque $x \in E$, on associe un élément de F noté $f(x)$
\end{definition}
\end{definitionbox}

\begin{definitionbox}
    \begin{definition}[Injectivité]
	Soit $f:E \to F$, on dit que $f$ est injective si pour chaque élément de F, il y a au plus un élément de E qui y est associé. Autrement dit :
	\begin{align*}
		\text{f injective} \iff \{\forall (x_1, x_2) \in E^2, f(x_1) = f(x_2) \implies x_1 = x_2\}
	\end{align*}
\end{definition}
\end{definitionbox}

\begin{definitionbox}
    \begin{definition}[Surjectivité]
	Soit $f:E \to F$, on dit que $f$ est surjective si pour chaque élément de F, il y a au moins un élément de E qui y est associé.
	Autrement dit :
	\begin{align*}
		\text{f surjective} \iff \{\forall y \in F, \exists x \in E, y = f(x)\}
	\end{align*}
\end{definition}
\end{definitionbox}

\begin{definitionbox}
    \begin{definition}[Bijectivité]
	Soit $f:E \to F$, on dit que $f$ est bijective si elle est injective et surjective, c'est-à-dire que pour chaque élément de F, il y a exactement un élément de E qui y est associé.
	Autrement dit :
	\begin{align*}
		\text{f bijective} \iff \{\forall y \in F, \exists! x \in E, y = f(x)\}
	\end{align*}
\end{definition}
\end{definitionbox}

\begin{definitionbox}
    \begin{definition}[Ensemble fini]
    Un ensemble E est un ensemble fini non-vide si et seulement si pour tout entier $n \geq 1$, il existe une application bijective de $\{1, 2, \ldots, n\}$ dans E.
\end{definition}
\end{definitionbox}

\begin{definitionbox}
    \begin{definition}[Fonction réciproque]
	Soient E et F deux ensembles. Supposons que $f:E \to F$ est une application bijective. On peut définir l'application 
	\begin{align*}
		f^{-1} : 
		\begin{cases}
			F &\to E \\
			y &\mapsto x
		\end{cases}
	\end{align*}
	comme étant la réciproque de $f$.
\end{definition}
\end{definitionbox}

\begin{definitionbox}
    \begin{definition}[Composition]
	Soient $f$ et $g$ deux applications telles que :
	$f:E \to F$ et $g:F \to G$ on a l'application $g \circ f : E \to G$ qui est définie comme étant la composée de $f$ et de $g$.
\end{definition}
\end{definitionbox}

\begin{definitionbox}
    \begin{definition}[Image directe et image réciproque]
	Soient $f:E \to F$ une application, A une partie de E et B une partie de F. Nous avons :
	\begin{align*}
		f(A) = \{f(x), x \in A\} &\text{ : image directe} \\
		f^{-1}(B) = \{x \in E, f(x) \in B\} &\text{ : image réciproque}
	\end{align*}
\end{definition}
\end{definitionbox}

\begin{propositionbox}
    \begin{proposition}[Propriétés sur les images directes et réciproques]
	Soient $f:E \to F$ une application et A, B des parties de F.
	\begin{enumerate}
		\item \label{prop_img_1} $f^{-1}(F \backslash A) = E \backslash f^{-1}(A)$
		\item \label{prop_img_2} $f^{-1}(A \cup B) = f^{-1}(A) \cup f^{-1}(B)$
		\item \label{prop_img_3} $f^{-1}(A \cap B) = f^{-1}(A) \cap f^{-1}(B)$
		\item \label{prop_img_4} $f(A \cup B) = f(A) \cup f(B)$
		\item \label{prop_img_5} $f(A \cap B) \subset f(A) \cap f(B)$
	\end{enumerate}
\end{proposition}
\end{propositionbox}

\begin{proof}
	\ref{prop_img_5} : $f(A \cap B) \subset f(A) \cap f(B)$ \\
	\par \noindent Soit $y \in f(A \cap B) = \{f(x), x \in A \cap B\}$, par définition : $\exists x \in A \cap B, y = f(x)$
	\begin{align*}
		x \in A \cap B &\iff x \in A \wedge x \in B \\
		x \in A &\implies y = f(x) \subset f(A) \\
		x \in B &\implies y = f(x) \subset f(B) 
	\end{align*}
	d'où $y \in f(A) \cap f(B)$
\end{proof} 

\begin{leftstroke}
\begin{remarque}
	\begin{align*}
		f(A \cap B) \neq f(A) \cap f(B)
	\end{align*}
\end{remarque}
\end{leftstroke}

	\chapter{Logique}
\begin{graybox}
	\begin{definition}[Assertion]
		\par Une \textbf{assertion} est une affirmation mathématique qui peut être vraie pou fausse.
	\end{definition}
\end{graybox}

\begin{graybox}
	\begin{definition}[Prédicat]
		\par Un \textbf{prédicat} est une "assertion" dépendant d'une ou plusieurs variables.
	\end{definition}
\end{graybox}

\begin{exemple}~
\begin{itemize}
\item "Tous les entiers sont des nombres rationnels" est une assertion.
\item "L'entier n est pair" est un prédicat.
\item "Le réel $x$ est le carré d'un nombre réel" est un prédicat.
\end{itemize}
\end{exemple}
\section{Opérations sur les prédicats}
\begin{table}[h!]
\centering
\begin{tabular}{|c|c|c|c|c|c|}
\hline
P & Q & P et Q & P ou Q & non(P) & P $\implies$ Q \\
\hline
V & V & V & V & F & V \\
\hline
V & F & F & V & F & F \\
\hline
F & V & F & V & V & V \\
\hline 
F & F & F & F & V & V \\
\hline
\end{tabular}
\end{table}

\subsection{Négation}
\begin{enumerate}
\item $P \implies Q$ est équivalent à non(P) ou Q
\item non(P ou Q) est équivalent à non(P) et non(Q)
\item non(P et Q) est équivalent à non(P) ou non(Q)
\end{enumerate}

\begin{remarque}~
\begin{enumerate}
\item Pour contredire "tous les éléments de E ont une propriété P", il suffit de trouver un contre-exemple

\begin{align*}
    non(\forall x \in E, P(x)) \iff \exists x \in E, non(P(x))
\end{align*}   

\item Pour contredire "il existe un élément de E vérifiant une propriété P", il faut montrer que tous les éléments de E ne vérifient pas la propriété P.

\begin{align*}
non(\exists x \in E, P(x)) \iff \forall x \in E, non(P(x))
\end{align*}

\item Une affirmation de type :
\begin{align*}
\exists ! x \in E, P(x) \iff 
\begin{cases}
& \exists x \in E, P(x) \\
& \text{Si } P(x) \text{ et } P(y) \text{ sont vrais, alors } x = y
\end{cases}
\end{align*}
\end{enumerate}
\end{remarque}

\begin{remarque}~
\\
    $\{ (a_n) \}_{n \in \N} \subset \R, \displaystyle \lim_{n \to +\infty} a_n = \alpha \in \R$ \\
A. Cauchy : 
\begin{align*}
    \forall \varepsilon > 0, \exists N, \left|a_n - \alpha \right| < \varepsilon, \forall n \geq N
\end{align*}
\end{remarque}


	\chapter{Nombres complexes}
\begin{align*}
(\N, +, \times) \subset (\Z, +, \times) \subset (\Q, +, \times) \subset (\R, +, \times) \subset (\C, +, \times)
\end{align*}
\par L'ensemble des nombres complexes est adapté à la résolution des équations algébriques.

\section{Vision algébrique des nombres complexes}

\begin{definition}{Forme algébrique des nombres complexes}
\begin{align*}
\C = \{a + ib \ | \ (a, b) \in \R^2 \}, \text{ avec } i = \sqrt{-1}
\end{align*}
\end{definition}

\begin{proposition}{Opérations sur les nombres complexes}
\begin{enumerate}
\item Somme : Soient $z = a + ib \in \C, w = c + id \in \C, (a, b, c, d) \in \R^4$
\begin{align*}
z + w = a + c + i (b + d)
\end{align*}

\begin{enumerate}
\item Associativité : $(z_1 + z_2) + z_3 = z_1 + (z_2 + z_3), (z_1, z_2, z_3) \in \C^3$
\item Elément neutre : $0 = 0 + i0 \implies z + 0 = 0 + z = 0, z \in \C$
\item Symétrique : $\forall z \in \C, \exists z', z + z' = z' + z = 0, z' = -z$
\begin{align*}
z = a + ib \implies -z = -a + i(-b)
\end{align*}
\item Commutativité : $z + w = w + z$
\end{enumerate}

\item Produit : Soient $z = a + ib \in \C, w = c + id \in \C, (a, b, c, d) \in \R^4$
\begin{align*}
z \cdot w = (ac - bd) + i(ad + bc)
\end{align*}

\begin{proof}
\begin{align*}
z \cdot w &= (a + ib) \cdot (c + id) \\
          &\mlqq=\mrqq a \cdot (c + id) + ib \cdot (c + id) \\
          &\mlqq=\mrqq a \cdot c + a \cdot id + ib \cdot c + ib \cdot id \\
          &\mlqq=\mrqq ac + i(ad) + i(bc) + i^2 bd \\
          &\mlqq=\mrqq ac - bd + i(ad + bc)
\end{align*}
\end{proof}

\begin{enumerate}
    \item Associativité :
    \begin{align*}
    (z_1 \cdot z_2) \cdot z_3 = z_1 \cdot (z_2 \cdot z_3), \forall (z_1, z_2, z_3) \in \C^3
    \end{align*}
    \item Elément neutre :
        \begin{align*}
        1 = 1 + i0 \implies z \times 1 = 1 \times z = z \\
        \forall z \in \C \backslash \{0\}, \exists z' \in \C, z \cdot z' = z' \cdot z =1
        \end{align*}
    \item Commutativité :
        \begin{align*}
        z \cdot w = w \cdot z, \forall (z, w) \in \C^2
        \end{align*}
    \item Distributivité :
        \begin{align*}
        &(z_1 + z_2) \cdot w = z_1 \cdot w + z_2 \cdot w \\
        &z \cdot (w_1 + w_2) = z \cdot w_1 + z \cdot w_2 \\
        &\forall (z, z_1, z_2, w, w_1, w_1) \in \C^6
        \end{align*}
\end{enumerate}

\end{enumerate}
\end{proposition}

\begin{remarque}
    $(\C, +, \times)$ est un corps commutatif 
\end{remarque}

\begin{definition}{Conjugué d'un nombre complexe}
    Soit $z = a + ib$ un nombre complexe, le nombre $\overline{z} = a - ib$ est dit le conjugué de z.
\end{definition}

\begin{proposition}{}
    Soient $z = a + ib, z'= a - ib, (a, b) \in \R^2, z \in \C$
    \begin{align*}
        z \cdot z' = a^2 + b^2
    \end{align*}
\end{proposition}
\begin{proof}
        \begin{align*}
            z \cdot z' &= (a + ib)(a - ib) \\
                        &= a^2 -iab + iab -i^2b^2 \\
                        &= a^2 + b^2
        \end{align*}
    \end{proof}


\begin{definition}{Module d'un nombre complexe}
    Soit $z = a + ib$ un nombre complexe, on définit son module comme étant :
    \begin{align*}
        |z| = \sqrt{a^2 + b^2}
    \end{align*}
\end{definition}

\begin{proposition}{Propriétés des modules}
    Soient $z = a + ib$ et $z = a' + ib'$ des nombres complexes, on a les propriétés suivantes sur les modules :
    \begin{itemize}
        \item $|z \cdot z'| = |z| \cdot |z'|$
        \item $\left| \frac{z}{z'} \right| = \frac{|z|}{|z'|}$
        \item $|z + z'| \leq |z| + |z'|$
        \item $|z|^2 = z \cdot \overline{z} = a^2 + b^2$
        \item $|z| \geq 0$
        \item $|z| = 0 \iff z = 0$
        \item $|z| = |\overline{z}| = |-z| = |-\overline{z}|$
    \end{itemize}
\end{proposition}

\begin{definition}{Partie réelle et partie imaginaire}
    Soit $z = a + ib \in \C, (a, b) \in \R^2$
    \begin{align*}
        &\Re(z) = Re(z) = a \text{ (Partie réelle)} \\
        &\Im(z) = Im(z) = b \text{ (Partie imaginaire)}
    \end{align*}
\end{definition}

\begin{proposition}{Propriétés}
    \begin{itemize}
        \item $z + \overline{z} = (a + ib) + (a - ib) = 2a \implies \Re(z) = \dfrac{z + \overline{z}}{2}$
        \item $z - \overline{z} = (a + ib) - (a - ib) = 2ib \implies \Im(z) = \dfrac{z - \overline{z}}{2i}$
    \end{itemize}
\end{proposition}

\section{Vision géométrique des nombres complexes}

\begin{definition}{Argument d'un nombre complexe}
    Soit $z$ un nombre complexe, l'argument de $z$, noté $\arg{(z)}$ représente l'angle entre la droite des réels et celle issue de l'origine et passant par $z$.
\end{definition}

\begin{proposition}{Propriétés des arguments}
    Soient $z, z_1, z_2 \in \C^3, n \in \N$
    \begin{itemize}
        \item $\arg{(z_1 \cdot z_2)} = \arg{z_1} + \arg{z_2}$
        \item $\arg{z^n} = n\arg{z}$
        \item $\arg{\frac{z_1}{z_2}} = \arg{z_1} - \arg{z_2}$
        \item $\arg{\frac{1}{z}} = -\arg{z}$
    \end{itemize}
\end{proposition}

\begin{definition}{Forme trigonométrique d'un nombre complexe}
    Soit $z$ un nombre complexe, on peut l'écrire sous sa forme trigonométrique ainsi :
    \begin{align*}
        z = r (\cos{(\theta)} + i \sin{(\theta)})
    \end{align*}
    Avec :
    \begin{itemize}
        \item $r = |z|$
        \item $\theta = \arg{(z)}$
    \end{itemize}
\end{definition}

\begin{proposition}{}
    Soient $z_1 = r_1(\cos{(\theta_1)} + i \sin{(\theta_1)})$ et $z_2 = r_2(\cos{(\theta_2)} + i\sin{(\theta_2)})$, deux nombres complexes. Nous avons la propriété suivante :
    \begin{align*}
        z_1 z_2 = r_1r_2(\cos{(\theta_1 + \theta_2)} + i\sin{(\theta_1 + \theta_2)})
    \end{align*}

\end{proposition}
\begin{proof}
        \begin{align*}
            z_1 z_2 &= (r_1(\cos{(\theta_1)} + i \sin{(\theta_1)}) (r_2(\cos{(\theta_2)}+i \sin{(\theta_2)}) \\
                    &=(r_1\cos{\theta_1} + ir_1 \sin{\theta_1}) (r_2\cos{\theta_2} + ir_2 \sin{\theta_2}) \\
                    &= (r_1\cos{\theta_1} \cdot r_2\cos{\theta_2}) + (r_1\cos{\theta_1} \cdot ir_2\sin{\theta_2}) + (ir_1\sin{\theta_1} \cdot r_2\cos{\theta_2}) + (ir_1\cos{\theta_1} + ir_2\sin{\theta_2})  \\
                    &= (r_1\cos{\theta_1})(r_2\cos{\theta_2}) - (r_1\sin{\theta_1})(r_2\sin{\theta_2}) + i((r_1\cos{\theta_1})(r_2\sin{\theta_2}) + (r_1\sin{\theta_1})(r_2\cos{\theta_2})) \\
                    &= r_1r_2((\cos{\theta_1} \cos{\theta_2} - \sin{\theta_1} \sin{\theta_2}) + i(\sin{\theta_1}\cos{\theta_2} + \cos{\theta_1}\sin{\theta_2})) \\
                    &= r_1r_2(\cos{(\theta_1 + \theta_2)} + i\sin{(\theta_1 + \theta_2)})
        \end{align*}
    \end{proof}

\begin{proposition}{Formule de Moivre}
    \begin{align*}
        (\cos{\theta} + i \sin{\theta})^n = \cos{(n\theta)} + i\sin{(n\theta)}
    \end{align*}
\end{proposition}

\begin{definition}{Forme exponentielle d'un nombre complexe}
    On peut écrire un nombre complexe sous une forme exponentielle :
    \begin{align*}
        z = r(\cos{\theta} + i\sin{\theta}) = re^{i\theta} 
    \end{align*}
\end{definition}

\begin{proposition}{Identité d'Euler}
    \begin{align*}
        e^{i \pi} = -1
    \end{align*}
\end{proposition}

\begin{proposition}{Formules d'Euler}
    \begin{align*}
        \cos{\theta} &= \frac{e^{i\theta} + e^{-i\theta}}{2} & \sin{\theta} &= \frac{e^{i\theta} - e^{-i\theta}}{2i}
    \end{align*}
\end{proposition}
\begin{proof}
        \begin{align*}
            e^{i\theta} + e^{-i\theta} &= (\cos{\theta} + i\sin{\theta}) + (\cos{\theta} - i\sin{\theta}) \\
                                       &= 2\cos{\theta} \\
        \frac{e^{i\theta} + e^{-i\theta}}{2} &= \cos{\theta}
        \end{align*}
        
        \begin{align*}
            e^{i\theta} - e^{-i\theta} &= (\cos{\theta} + i\sin{\theta}) - (\cos{\theta} - i\sin{\theta}) \\
                                       &= 2i\sin{\theta} \\
            \frac{e^{i\theta} - e^{-i\theta}}{2i} &= \sin{\theta} 
        \end{align*}
\end{proof}

\begin{remarque}[Passer de la forme algébrique à la forme trigonométrique]~ 
    \\
    Soit $z = a + ib, (a, b) \in \R^2$ un nombre complexe sous sa forme algébrique, on peut passer sous la forme trigonométrique ainsi :
    \begin{align*}
        \cos{\theta} &= \frac{a}{|z|} & \sin{\theta} &= \frac{b}{|z|}
    \end{align*}
    \begin{exemple}
        $z = 1 + i$
        \\
        On a : $|z| = \sqrt{1^2 + 1^2}$
        \\
        On a donc :
        \begin{align*}
            \cos{\theta} &= \frac{1}{\sqrt{2}} & \sin{\theta} &= \frac{1}{\sqrt{2}} \\
                         &= \frac{\sqrt{2}}{2} &              &= \frac{\sqrt{2}}{2}
        \end{align*}
        On en déduit donc que $\theta = \frac{\pi}{4}$. \\
        Ainsi $z = \sqrt{2}\left(\cos{\frac{\pi}{4}} + i\sin{\frac{\pi}{4}}\right)$
    \end{exemple}
\end{remarque}

\begin{definition}{Racine n-ième d'un nombre complexe}
    Soit $z \in \C$. On appelle racine n-ième du nombre complexe $z$ tout nombre complexe $w \in \C$ vérifiant :
    \begin{align*}
        w^n = z
    \end{align*}
    Un complexe non nul $z = \rho e^{i\theta}$ admet n racines n-ièmes données par :
    \begin{align*}
        Z_k = \rho^{\frac{1}{n}} e^{i\left(\frac{\theta}{n} + \frac{2k\pi}{n} \right)} 
    \end{align*}
\end{definition}

\begin{definition}{Racine n-ième de l'unité}
    On appelle racine n-ième de l'unité, une racine n-ième de 1, on notera $\mathbb{U}_n$ l'ensemble des racines n-ièmes de l'unité :
    \begin{align*}
        \mathbb{U}_n = \{z \in \C | z^n = 1 \}
    \end{align*}
    Les racines n-ièmes de l'unité sont de la forme :
    \begin{align*}
        \omega_k = e^{\frac{2ik\pi}{n}}, k \in \llbracket 0, n - 1 \rrbracket
    \end{align*}
\end{definition}

\section{Géométrie des nombres complexes}
\begin{itemize}
    \item $z \mapsto z + a, (a \in \C)$ : translation de vecteur $\vect{u}$ d'affixe $a$
    \item $z \mapsto az, (a \in \R^*)$ : homothétie de rapport $a$
    \item $z \mapsto e^{i\theta}z, (\theta \in \R)$ : rotation d'angle $\theta$ et de centre $0$
    \item $z \mapsto \overline{z}$ : réflexion par rapport à l'axe des réels
    \item $z \mapsto a + e^{i\theta} (z - a)$ : rotation d'angle $\theta$ de centre $a$
    \item $z \mapsto e^{2i\theta} \cdot \overline{z}$ : réflexion par rapport à la droite formant un angle $\theta$ avec l'axe des réels.
\end{itemize}

\subsection{Equation d'une droite}
\begin{itemize}
    \item L'axe des réels : $\overline{z} = z$
    \item Un axe formant un angle $\theta$ avec l'axe des réels : $\overline{e^{-i\theta}z}= e^{-i\theta}z$
    \item L'asymptote verticale de partie réelle $a$ : $z + \overline{z} = 2a$
\end{itemize}

\begin{exemple}
    $z \mapsto \frac{1}{z}$ 
    \\
    On pose : $w = \frac{1}{z}$
    \\
    On a donc : $z = \frac{1}{w}$
    \\
    $z + \overline{z} = 2 \implies \frac{1}{w} + \overline{\frac{1}{w}} = 2 \implies \frac{1}{w} + \frac{1}{\overline{w}} = 2$ \\
    $w\overline{w} \left( \frac{1}{w} + \frac{1}{\overline{w}} \right) = 2w\overline{w}$
    \\
    On a donc : $\overline{w} + w = 2w\overline{w} \implies 2w\overline{w} - w - \overline{w} = 0$ 
    \\
    C'est à dire : $w\overline{w} - \frac{1}{2}w - \frac{1}{2}\overline{w} = \left( w - \frac{1}{2} \right) \left( \overline{w} - \frac{1}{2} \right) - \frac{1}{4} = 0$
    \\
    Ce qui équivaut à $\left| w - \frac{1}{2} \right|^2 = \left( \frac{1}{2} \right)^2 \iff \left|w - \frac{1}{2}\right| = \left(\frac{1}{2}\right)$
\end{exemple}

\begin{exemple}
    $P = \left\{ z \in \C, \Im{(z)} > 0\right\}$ : le demi-plan de Poincaré    
    \\
    Déterminer l'image de $P$ par la transformation $z \mapsto \frac{z - i}{z + i}$
    \\
    \begin{enumerate}
        \item $w = \frac{z - i}{z + i}$, exprimer $z$ en fonction de $w$.
            \begin{align*}
                &w = \frac{z - i}{z + i} \\
                \iff &w(z + i) = z - i \\
                \iff &w(z + i) + i = z \\
                \iff &wz + wi + i = z \\
                \iff &wz - z = -wi - i \\
                \iff &z(w + 1) = -wi - i \\
                \iff &z = \frac{-wi - i}{w + 1} \\
                \iff &z = \frac{-i(w + 1)}{w + 1} 
            \end{align*}
        \item $z \in P \iff \Im{(z)} > 0$ 
            \\
            $z = x + iy$, $\overline{z} = x - iy$, on a : $z - \overline{z} = 2iy$ \\
            Si on a $\Im{(z)} = y > 0 \iff \frac{1}{2i}(z - \overline{z}) > 0$
            \\
            A la fin on obtient : $w\overline{w} < 1 \implies |w| < 1$
    \end{enumerate}
\end{exemple}

	

	
	\bibliographystyle{plain}
	\bibliography{refs} 
\end{document}
